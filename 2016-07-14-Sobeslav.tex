% Zdroj: https://www.idnes.cz/zpravy/sto-pohledu/100-pohledu-na-cesko.A160714_092612_domaci_jj/diskuse
% Datum: 14. 7. 2016 12:29
% Foto: https://1gr.cz/fotky/idnes/16/063/org/JB643b8e_00_V2010.jpg

Nalevo od tohoto kostelíka býval starý mostecký hřbitov. V letech
1967-70 došlo k jeho likvidaci. Vzpomínám, jak jsem někdy v té době
kolem něj procházel v pozdních večerních hodinách k domovu, když mě
oslovily dvě postavy sedící na hřbitovní zdi.

Hrozně jsem se lekl, protože před tím jsem si jich vůbec nevšiml. Krom
toho, že byla tma, tak jejich vyzáblé postavy kryla z části koruna
hustého kaštanu.

"Koukám, že ty ses moc vysokého věku nedožil, chlapče," povídá jedna z
těch postav. "V kterém roce jsi vlastně umřel?"

Po prvotním šoku jsem odpověděl po pravdě, že jsem ještě neumřel a že
jdu od kamaráda u kterého jsme hráli "Člověče nezlob se". Na další
otázky jsem už ale nehodlal čekat a jen co jsem je minul, přešel jsem
do běhu takové intenzity, že jsem se snad ani nedotýkal země. Cestou
jsem předběhl tramvaj, která mi předtím ujela.

Zastavil jsem se až doma a už v klidu si přemáchl spodní prádlo. Sice
nevím kdo to byl, ale tuhle trasu už jsem za tmy nikdy neabsolvoval.

Dnes, kdykoli sleduji v televizi atletické závody a vidím, jak diváci
oslavují Usaina Bolta, je mi malinko smutno, když žádný z nich netuší,
že srovnatelných výkonů bylo dosaženo na mostecku již několik
desetiletí předtím.
