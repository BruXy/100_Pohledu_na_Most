% Zdroj: https://www.idnes.cz/zpravy/sto-pohledu/100-pohledu-na-cesko.A161212_083740_domaci_hro/diskuse/4
% Datum: 12. 12. 2016 10:14
% Foto: https://1gr.cz/fotky/idnes/16/113/org/JB678737_bbe.jpg
% Body: +120 -6

Městské divadlo v~Mostě. Ovšem pro fundovanější informace k~danému
snímku jsem se přesto rozhodl vyrazit za historikem Stříklým, kterého
jsem poznal když jsem mu pomáhal na nohy poté, co jej srazil jeden
z~přeplněných nákupních vozíků v~místním supermarketu. Bylo to tam ten
den vážně o~život. Ostatně jako pokaždé, když se sejde nový akční
leták s~vyplácením sociálních dávek. Stříklý se nedlouho před tím
vrátil z~Thajska, kde nějaký čas studoval pracovní návyky tamních
prostitutů. Že je odborníkem na slovo vzatým, mi dokázal hned po
několika minutách zkoumání fotografie, o~níž prohlásil že byla
pořízena v~čase 17.25 mezi léty 1850-1950 a bylo po dešti. Další
detaily jsem se dozvěděl, až když mi umožnil nahlédnout do svých
historických análů. Jeho zájem o~průzkum mého análu, jsem důrazně
odmítl a v~té době to ještě považoval za vtip. Ale zpátky k~obrázku.
Podařilo se mi zjistit, že snímek pochází ze seznamky víkendového
vydání Brüxer Zeitung. Zajímavostí je, že byl použit u~inzerátů dvou
různých osob a to nezávisle na sobě. A~to u~té dámy v~popředí
s~deštníkem a dívky v~laškovné poloze na schodech divadla. Přesným
zněním inzerátů nemohu bohužel sloužit. Řešil jsem to použitím
překladače a navíc se kolem mě začal producírovat Stříklý v~nějakém
směšném erotickém prádle. Ale u~dámy snad stálo něco jako: "Žena
v~nejlepších letech, mladšího vzhledu. Ne vlastní vinou 6x ovdovělá,
hledá muže za účelem sňatku, který má vše a rád by se o~to s~někým
podělil. Zn. Pokročilý věk, či smrtelná nemoc nejsou překážkou".
A~u~té dívky to bylo asi nějak takhle: "Půvabná mladá dívka tě ráda uvítá
ve své komůrce. Přírodní trojky, dole upravená. Zn. Bez peněz, na mě
nelez". Na víc už nebyl čas, protože když začal kolem mě kroužit
Stříklý s~bzučícím vibrátorem došlo mi, že je nejvyšší čas vypadnout.
Což jsem i přes jeho odpor rázně učinil ale pro jistotu jsem se
obouval až před domem. Teda ale, mít v~tu chvíli po ruce plně naložený
nákupní vozík, s~chutí bych ho přejel. A~snad ještě couvnul.

Něco k~historii toho půlkoně na průčelí by nebylo?

Domnívám se, že onen znak tam byl umístěn jako pocta mosteckému
perníkářskému cechu, který přispěl nemalou finanční částkou na stavbu
divadla v~roce 1911. Navíc jedním z~menších přispěvatelů byl i hřebčín
v~nedalekých Svinčicích a tak lze říci, ze zde byli trefeni dva koně
jednou půlkou.


% Stanislav Falout
% https://www.idnes.cz/zpravy/sto-pohledu/100-pohledu-na-cesko.A161212_083740_domaci_hro/diskuse/1
% Je ostudou seriózního serveru iDnes, že dovolí na svém diskuzním fóru
% zveřejňovat lživé a pavědecké nesmysly různých přemoudřelců,zejména
% jistého V. Beneše. Toto individuum, vydávající se za odborníkana
% historii, zde opakovaně publikuje výplody své fantazie, které
% nemajínejmenší oporu ve faktech.


