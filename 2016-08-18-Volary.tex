% Zdroj: https://www.idnes.cz/zpravy/sto-pohledu/100-pohledu-na-cesko.A160818_093700_domaci_jj/diskuse
% Datum: 18. 8. 2016 10:41
% Foto: https://1gr.cz/fotky/idnes/16/082/org/MBB655a1d_sta.jpg

Historický snímek hornické kolonie na okraji Mostu. Jedná se o tehdy
poměrně frekventovanou Volskou ulici. Fotografovi se podařilo zachytit
kromě těch dvou volů v popředí i jiné zajímavé postavy. Že byl snímek
pořízen právě v době probíhající dopravní špičky, svědčí další
přijíždějící volský potah. Ten je však vzhledem k dodržování
předepsaného odstupu mezi volskými potahy vidět již jen matně. V tomto
případě se nejspíš jedná o jednostopé vozidlo s nižší kubaturou a tedy
i spotřebou. Dopravní policista sedící vpravo na kládách s vysílačkou
a se sluchátky na uších, zřejmě konzultuje nastalou dopravní situaci
se svými kolegy na dalších úsecích. Je dost možné, že četnost volů na
snímku by mohla některé tipující mylně přivést na myšlenku, že se
jedná o Volyni, případně o Volary. Budova vpravo je zájezdní hostinec
U lucerny. V nočních hodinách pak U červené lucerny. Paní hostinská
čilý ruch na Volské ulici sleduje s rukama v bok. Muž vlevo, kterého
před chvílí vyhodila dveřmi, uvažuje o návratu oknem. Hostinec
U~lucerny hojně navštěvovali právě řidiči volských potahů a především
ti z MDVD (Mezinárodní dálkové volské dopravy), kteří ho využívali k
dodržování předepsaných bezpečnostních přestávek. Podle prázdného
parkoviště před hospodou, to však v danou chvíli na nějakou tlačenici
uvnitř nevypadá. Proto se zdá, jakoby paní hostinská očima přímo
hypnotizovala právě projíždějící volský povoz a doufá, že snad hodí
pravý blinkr. Ale podle postavení šoférova biče tomu nic nenasvědčuje.
Tak snad až ten další, paní hostinská... Ti dva chlapci v kloboucích,
jsou zřejmě synové paní hostinské, kteří se starali o před hospodou
zaparkované voly. O ty uvnitř, už se postarala paní hostinská sama.
Bosý, zády otočený chlapec bez klobouku kráčející vpravo, je
pravděpodobně děvče. Lucerna na rohu hostince svítívala v noci červeně
a bílá košile ledabyle přehozená v prvním patře přes šňůru a patřící
dceři paní hostinské signalizovala, že dneska by to šlo.
