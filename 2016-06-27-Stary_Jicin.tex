% Zdroj: https://www.idnes.cz/zpravy/sto-pohledu/100-pohledu-na-cesko.A160627_091354_domaci_hro/diskuse/1
% Datum: 27. 6. 2016 11:00
% Foto: https://1gr.cz/fotky/idnes/16/063/org/JB641402_xL3101inn.jpg
% Body: +95 -1

\chapter{Chrámce na Mostecku}

Na obrázku vidíme sklizeň brambor v~obci Chrámce na Mostecku. A~to více jak
před sto lety. Vrch Číčov byl tehdy ještě zalesněný a stála na něm zřícenina
hradu. To obojí tam dnes chybí. V~současnosti je obec vyhlášená především
pěstováním vinné révy a výrobou vína. Nejslavnější víno z~tamní produkce
Chrámecký Divoch, je pro obyvatele okresu pojmem. Název zřejmě vznikl spojením
místa původu a chováním mostečanů po jeho konzumaci. To děvčátko na fotografii
je pravděpodobně babička strejdy Emana, který z~Chrámců pocházel. No,
strejdy\dots

Jako dneska vzpomínám na den, kdy k~nám domů přivedla matka jednoho chlápka a
povídá: \uv{Tohle je pan Eman. Pracuje u~nás ve fabrice a bude u~nás bydlet.
Můžeš mu říkat tati.}

\uv{Ale já už tátu přece mám, sedí naproti v~hospodě,} namítl jsem.

\uv{Ten, co ho máš na mysli sice sedí, ale úplně někde jinde a bude tam
dlouho,}pokračovala. Později jsem se dozvěděl, že otce zavřeli za to, že
protestoval proti vstupu vojsk Varšavské smlouvy v~srpnu 1968 tím, že na mě
přestal posílat alimenty.

\uv{Prozatím můžeš panu Emanovi říkat strejdo,} navrhla matka. \uv{Ale strejdu
mám už taky, Karla přece,} odvětil jsem.

\uv{Toho vochlastu co má trvalý bydliště na záchytce, mi ani nepřipomínej.
Posledně když tu byl, ukrad mi ze špajzu bažanta, co tam visel aby odležel a
doma ho upek a sežral. Mně se pak snažil namluvit, že ho měl v~úmyslu oživit a
přimět k~odletu do Afriky, na pomoc hladovějícím. A~nakonec strejdů můžeš mít i
víc, jiný to maj taky tak,} uzavřela konverzaci matka.

A~tak jsem měl strejdu Emana a nijak moc to neřešil. Když pak jednoho dne odjel
na matčině dámském kole a s~ním zmizelo i několik věcí co měly nějakou cenu,
bylo mi jasné, že sami dlouho nezůstaneme.
