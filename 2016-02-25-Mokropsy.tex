% Zdroj: https://www.idnes.cz/zpravy/sto-pohledu/100-pohledu-na-cesko.A160224_151226_domaci_jav/diskuse/2
% Datum: 25. 2. 2016 9:25
% Foto: https://1gr.cz/fotky/idnes/16/013/org/JB60e946_syV4C15.jpg
% Body: +15 -4

\def\poznamka{Z fotografie to není patrné, ale pravděpodobně by mohlo jít i
o~krakorec korečkového RK\,5000. A~než o~zbytky pásového dopravníku se bude
jednat nejspíš o~stádo muflonů, kterým se v~této oblasti mimodřáně dařilo,
protože jejich střílení bylo v~rámci povrchových dolů od roku 1897 přísně
zakázáno poté, co si jeden z~myslivců spletl štajgra Vodrážku s~jedním z~nich.}

\chapter{Zřícený most přes řeku Bílinu}

%Je to město Most, nikoli Brno, jak se zde bude někdo domnívat.

Na obrázku je zřetelně vidět, jak už před více než sto lety, bylo toto město
nuceno ustupovat povrchové těžbě hnědého uhlí. Na vytěžené planině jsou ještě
zbytky pásových dopravníků a za horizontem je zřetelně vidět vrchní část
velkostroje. Pravděpodobně KU\,800\footnote{\poznamka}. Zřícený most přes řeku
Bílinu, je důsledkem přetěžování vagonů po okraj naplněných uhlím.

