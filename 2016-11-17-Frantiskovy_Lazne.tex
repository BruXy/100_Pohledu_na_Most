% Zdroj: https://www.idnes.cz/zpravy/sto-pohledu/100-pohledu-na-cesko.A161116_095436_domaci_jj/diskuse/3
% Datum: 17. 11. 2016 11:46
% Foto: https://1gr.cz/fotky/idnes/16/111/org/JB671652_col.jpg
% Body: +145 -0

Konec června roku 1901. Snímek byl pořízen v jednom z nákupních center
na lázeňské kolonádě v Mostě. Přesné datum pořízení fotografie je
obtížné zjistit, ale ten den bylo vedro a podle počtu nakupujících
bylo těsně před vejplatou. Snad si tedy řekněme něco málo o osobách
zachycených tomto na snímku. Začněme u dámy v popředí, která vedla
obchod s koženou a textilní galanterií jistého pana Barthla. Jmenovala
se Johanna Straussová a zdědila hudební nadání po svém nevlastním
otci, králi valčíků. Dokonce vyhrála konkurs do vídeňského
symfonického orchestru jako hráčka na triangl. Pak se ovšem přišlo na
to, že to bylo celý nějaký cinknutý a místo nakonec získal někdo jiný.
Krátce se sice ještě pokoušela zúročit svůj vrozený hudební talent
jako komponistka, ale koncertní představení její sonáty pro klavír a
cirkulárku na mosteckém 1. náměstí skončilo pro nedostatek dřeva a
různých částí horních končetin fiaskem. Druhá dáma na snímku se
jmenovala Líba Copperfieldová a provozovala krámek s domácími
potřebami a také potřebami pro iluzionisty. Právě do jednoho z nich se
zamilovala tak mocně, že úplně ztratila hlavu. Ale že to bylo až tak
doslova jak ukazuje snímek, by mě teda ani ve snu nenapadlo. O třetí
ženě na fotografii víme jen to, že se jmenovala Madlenka Joudová, měla
tři děti a živila se šitím a úpravou oděvů a módních doplňků. Ten její
Jouda byl zaměstnán jako dobrovolný hasič, ale protože v Mostě
hořívávalo spíš sporadicky, hasil hlavně žízeň. Jeho pití bývalo
příčinami manželských krizí a tak to počátkem roku 1912 vypadalo málem
na rozvod. Ale protože měla Madlenka zlaté srdce, nakonec mu odpustila
i to, že propil ty lístky na Titanic na které se dlouhá léta dřela.
Nakonec je tu muž sedící na lavičce před vietnamskou večerkou. Podle
všeho by mělo jít o bezdomovce Ferdinanda, který se už v mládí naučil
nedělat nic a po celý zbytek života se v tom ještě postupně
zdokonaloval.

