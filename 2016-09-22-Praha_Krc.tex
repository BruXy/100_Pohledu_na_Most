% Zdroj: https://www.idnes.cz/zpravy/sto-pohledu/sto-pohledu-na-cesko-nemocnice-krc.A160922_085108_domaci_hro/diskuse/4
% Datum: 22. 9. 2016 10:01
% Foto: https://1gr.cz/fotky/idnes/16/091/org/JB65d070_cO506.jpg

Ano, toto je opravdu poměrně nový snímek jednoho z mosteckých
panelových sídlišť. Podle výsypky v popředí byl zřejmě pořízen ze
zakladače ZP 6600, které se v okolí města i v současnosti vyskytují v
poměrně hojném počtu. Ta nízká budova vlevo je sportovní hala
Lokomotivy Most. V té jsem svého času navštěvoval tamní boxerský
oddíl. Matka mé nadšení pro tento druh pohybové aktivity pranic
nesdílela a tak jednou ještě před zavírací hodinou, vyzvedla otce z
jeho oblíbeného restauračního zařízení a přišli se na mě podívat. Asi
se jí to moc nezdálo, protože po tom co viděla, následovala její živá
diskuse s trenérem. Docela jí vadilo, že většina ostatních měla
rukavice od krve a jenom já obličej. Taky se trenéra ptala co je
pravdy na tom, že se po úderech do hlavy blbne. On jí odpověděl, že to
není žádná pravda a že já už jsem k nim do oddílu v tomhle stavu
přišel. Dokonce si prý všiml, že po několika úderech do hlavy
přestávám zadrhávat. Otec před návratem do hostince jen suše prohodil,
že když se mlátěj v hospodě židlema je u toho větší legrace a že až
budu větší, tak mě to taky naučí. K tomu už jaksi nedošlo, protože
jednoho dne zmizel a nebýt toho, že se u nás objevil někdo další,
nejspíš bych to ani nezaregistroval. Do boxu jsem přes matčiny výhrady
přesto ještě nějaký čas docházel, ale když mi v klinči jedna z holek
hlavou přerazila jedničku vlevo nahoře a já dostal zlatou korunku,
definitivně mi to zatrhla. Když jsem tuto skutečnost oznamoval
trenérovi, tvářil se spokojeně a prohlásil, že kdyby se vedly
statistiky prohraných zápasů K.O. v prvním kole, byl bych v tom
nejlepší na světě.


