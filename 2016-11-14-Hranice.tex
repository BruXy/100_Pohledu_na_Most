% Zdroj: https://www.idnes.cz/zpravy/sto-pohledu/100-pohledu-na-cesko.A161114_091540_zahranicni_jj/diskuse/2
% Datum: 14. 11. 2016 11:33
% Foto: https://1gr.cz/fotky/idnes/16/112/org/JB671656_reV10C24OK.jpg
% Body: +134 -1

Přiznám se, že tentokrát jsem opravdu dlouho váhal. Nakonec jsem se i
přes nápovědu, rozhodl pro Dolní Jiřetín, další ze zaniklých obcí
v~okrese Most. Na snímku mě sice zprvu zaráželo, jakoby tam oproti jiným
dobovým snímkům obce, jeden z~těch protilehlých kostelů přebýval. Ale
Lojza co u~nás na šachtě fotíval do závodních novin "Rudé uhlí" a taky
úderníky na nástěnku, mi dole před večerkou za lahváče vysvětlil, že
snímek byl pravděpodobně pořízen nějakým starým typem zrcadlovky a ty
že to občas dělávaly. Definitivně jsem se ale pro Dolní Jiřetín
rozhodl až po nahlédnutí do tamní kroniky. Ta byla klíčem pro
částečnou identifikaci chlapců zachycených na snímku. Ty fotograf
zastihl během lovu hrabošů polních. Ten mladší klečící, který právě
jednoho z~hrabošů svírá v~ruce, byl jakýsi Tonda Káně a byl v~tomto
oboru skutečným mistrem. Taky mu nikdo neřekl jinak, než Káně Myšilov.
O~stojícím chlapci, který už jen ulovené hlodavce napichuje na kovový
bodec jako na špíz, se kronika nezmiňuje. Hraboši nebyli určeni ke
konzumaci, ale majitel okolních polností sedlák Ruchadlo hochům podle
počtu ulovených kusů vyplácel menší finanční hotovost. Vzhledem
k~vyšším výnosům se mu to bohatě vyplácelo. O~Tondově životě víme
s~určitostí jen to, že byl mimořádně plodný. S~děvečkou Emilkou měl 4
děti a další na cestě, když byl po vypuknutí 1. světové války povolán
do armády. Tam se v~září 1914 při jednom z~ústupů na východní frontě
    udeřil do hlavy o~kovový zátaras a upadl do bezvědomí a krátce na
    to pak i do zajetí. Odtud putoval vzledem k~nedostatku mužů
    pracovat do jednoho z~ruských statků, později kolchozů, kde stihl
    mít do konce války v~roce 1918 s~několika tamními ženami dalších
    12-14 dětí. Za své přijal i lépe se hodící jméno Myšilov. Ovšem na
    onom statku, zřejmě kvůli narůstajícím konfliktům mezi ruskými
    ženami a problémům s~výživným na děti v~Československu, jeho stopa
    počátkem jara 1919 končí. Informace, že dožil obklopen mnoha dětmi
    v~Japonsku v~Osace, je nutno brát s~rezervou.

