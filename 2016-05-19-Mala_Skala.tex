% Zdroj: https://www.idnes.cz/zpravy/sto-pohledu/100-pohledu-na-cesko.A160512_123132_domaci_hro/diskuse/3
% Datum: 19. 5. 2016 10:29
% Foto:  https://1gr.cz/fotky/idnes/16/052/org/JB6337c7_L2102.jpg
% Body: +108 -4

\chapter{Nádraží v~Rudolicích}

Rudolice u~Mostu a dnes již neexistující nádraží. O~kousek dál bylo totiž
postaveno nové mostecké a na tom jsem už vystupoval při svém návratu z~vojny.
Když jsem dva roky před tím vojákovat odjížděl, bylo to ještě z~toho původního
ve starém Mostě. Republika byla tenkrát o~kousek delší a já vyfasoval
Michalovce. Dál už to snad ani nešlo. Z~Mostu nás tam tenkrát jelo možná dvacet
a více či méně zmoženi alkoholem, jsme nasedli do vlaku. Čekala nás dlouhá
cesta a tak si někteří více společensky unavení, vylezli nahoru do sítí pro
zavazadla.

Vojta, co seděl vedle mě, neustále pokukoval po hezké slečně, sedící naproti
němu. \uv{Člověče, co bych za to dal, kdyby mi taková padla do náručí,}
zašeptal mi do ucha. Napadlo mě, že to nebude až takový problém a odešel jsem
zatáhnout za záchrannou brzdu.

Při návratu do kupé jsem okamžitě pochopil, že se to zas tak úplně nepovedlo.
Vojta právě počítal zuby, které mu slečna vyrazila úderem hlavou. Asi jich bylo
víc, protože mu vůbec nebylo rozumět, jak šišlal. Dobře nevypadali ani ti, co
leželi v~sítích pro zavazadla na těch nesprávných stranách našeho i sousedních
kupé. Jeden si zlomil obě ruce a druhý tři žebra a klíční kost. Ti dostali
odklad a rukovali znovu o~půl roku později. Ten třetí, co spadl rovnou na
hlavu, dostal dokonce modrou knížku o~níž se u~odvodu usilovně, leč bezúspěšně
snažil. Po této příhodě již nebylo pochyb, že je tato ve správných rukou.

Když odvezli raněné, stále ještě probíhalo v~našem vagónu vyšetřování. Teprve
když jsem jim oznámil, že jsem viděl utíkat dvě podezřelé osoby od vlaku přes
pole k~lesu, začalo se vyšetřování ubírat jiným směrem a my konečně mohli
pokračovat v~cestě, jejímž cílem byla služba vlasti.
