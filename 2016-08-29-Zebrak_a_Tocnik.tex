% Zdroj: https://www.idnes.cz/zpravy/sto-pohledu/100-pohledu-na-cesko.A160829_104040_domaci_jav/diskuse/3
% Datum: 29. 8. 2016 11:52
% Foto: https://1gr.cz/fotky/idnes/16/081/org/JB652409_Kar.jpg

Tak Hněvín jsem poznal, ale že by byla na Resslu nějaká zřícenina, to
slyším prvně. Za mě už tam bylo letní kino a celý kopec zalesněný. My
chodili se školou do kina Svět, protože bylo blízko naší školy. V něm
tam takhle jednou během filmu naše soudružka učitelka okřikla Ferdu s
Karlem, kteří se hlasitě dohadovali o tom, k čemu všemu slouží ta malá
prdelka co jí mají děvčata vpředu. Hned na to ale na ní vystartovala
uvaděčka, která jí zřejmě kvůli menšímu vzrůstu v té tmě považovala za
jednu z nás, posvítila na ní baterkou a řekla jí at' okamžitě zavře
klapačku, nebo dostane pár facek. Soudružce učitelce to přišlo líto a
začala brečet. Při cestě z kina to pak odůvodňovala tím, že jí dojal
děj filmu. Konkrétně osud toho soudruha, co těžce nesl neplnění plánu
a jako na potvoru mu to přestalo klapat doma, protože byl pořád v
práci. Soudružka učitelka tenkrát měla docela den blbec. Při cestě z
kina si nás ani nespočítala a tak ani nezjistila, že chybí Kamil. Ten
zůstal v kině uskřípnutý mezi sedadlem a opěradlem hlavou dolů. To se
mu přihodilo, když se chtěl pokusit sledovat závěr filmu vzhůru nohama
a tak udělal na sedadle stojku. To se mu ale sklopilo a on nemohl ven.
Našla a osvobodila ho až uklízečka. Před tím mu ovšem stihla vycucnout
vysavačem klíče od bytu a nějaké drobné. Nakonec jsme tenkrát všichni
šťastně doputovali z kina domů. I~když někteří později. Kromě Kamila
ještě Honza, který zůstal zaklesnutý v kanále, do kterého mu spadla
koruna a kterého za nohy vytáhl až druhý náhodný kolemjdoucí. Ten
první náhodný kolemjdoucí mu jen vytáhl ze zadní kapsy kalhot
peněženku ve které měl 150 korun a které mu dala maminka na obědy do
družiny. Dokonce až po třech dnech přivezli nějací pánové v civilu
Edu, který si prý v kině zapomněl brýle a pak cestou domů zabloudil.
Nakonec měl veliké štěstí, že ho na poslední chvíli našli pohraničníci
až někde u Aše, jen několik desítek metrů od přechodu do západního
Německa.
