% Zdroj: https://www.idnes.cz/zpravy/sto-pohledu/100-pohledu-na-cesko.A160818_101737_domaci_jj/diskuse/2
% Datum: 22. 8. 2016 13:23
% Foto: https://1gr.cz/fotky/idnes/16/082/org/MBB655a2d_sta.jpg

Na snímku z přelomu 19. a 20. století vidíme část obce Souš u Mostu.
Mládenec na zachycený na snímku a překonávající rozvodněnou Řeku
Bílinu je ševcovský tovaryš Klaczek. Zda je zachycen na cestě do
práce, či jen tak zapózoval na druhém břehu stojícímu fotografovi, lze
už jen ztěží určit. Možná že obojí. Hrad na snímku je sice Hněvín,
ovšem je to jen jeden z návrhů, který výběrovým řízením neprošel. Ten
vítězný stojí na kopci na městem dodnes. V místech kde byla fotografie
pořízena je v současnosti autodrom. Dlouhá léta před tím se však
automobilové a motocyklové závody jezdily v ulicích města. Mostecké
závody byly hodně populární u našich přátel z NDR. Nejspíš proto, že
je taky skoro nikam jinam nepustili. Přehoupnout se přes Krušné hory
po tzv. "Trabantstrasse" pro ně nebyl žádný problém. A tehdy jsem se
já, v době hluboké totality, rozhodl uskutečnit svůj první
podnikatelský záměr. V papírnictví jsem po dvaceti korunách nakoupil
plakáty s vozy Formule 1 a srolované jsem se je vydal v den konání
závodů nabízet příznivcům automobilismu z východního Německa. "Když je
po třiceti korunách prodám, nějakou tu pětku to hodí," přemítal jsem
nahlas. Jenže když jsem se vmísil do davu přihlížejících, dostal jsem
strach. Co když mě u toho chytí někdo od SNB? To by byl malér. Nebo
dokonce od STB? Určitě jich tu bude v civilu plno. Nakonec mně
představa zmařené investice dodala odvahy a já rozbalil plakáty v
klubku pivem rozjařených Němců. Ti začali nadšeně hulákat a svolávat
své krajany ze širokého okolí. V tu chvíli už skoro nikdo nesledoval
dění na závodní dráze, zatímco já už jsem se viděl za katrem. A tak na
otázku "ví fíl króne", jsem vyhrkl "cén króne". Plakáty zmizely
rychlostí blesku a já skoro stejně rychle po nich. Nakonec jsme byli
spokojeni všichni. Němci nakoupili levně plakáty a mě nezavřeli za
nedovolené podnikání.

