% Zdroj: https://www.idnes.cz/zpravy/sto-pohledu/100-pohledu-na-cesko.A161128_092543_domaci_jj/diskuse/2
% Datum: 28. 11. 2016 10:53
% Foto: https://1gr.cz/fotky/idnes/16/113/org/JB67873d_yy.jpg
% Body: +139 -0

Že se jedná o~Lom u~Mostu mohu dnes napsat jen díky mravenčí práci
profesora Třaskavého a jeho týmu. Za což jim touto cestou děkuji. Na
druhou stranu nemohu přehlédnout skutečnost, že když už se u~nás ve
výčepu vožrali jako prasata, tak se k~tomu měli postavit jako chlapi a
ne zmizet bez placení jako malí kluci. Nyní k~obrázku. Určit místo
podle vzhledu domů na náměstí bylo nemožné, protože v~Lomu to vypadá
teda úplně jinak. Identifikovat muže v~dlouhém kabátě stojícího vpravo
u~domu, se nepodařilo ani tehdejším četníkům, kterými byl hledán na
základě hlášení několika místních žen a kterým se chlubil že pod tím
kabátem vůbec nic nemá. Sestavit portrét mužova obličeje se bohužel
ženám vůbec nedařilo. Naopak vzácná shoda panovala při popisu
velikosti a tvaru jeho přirození. Tato informace však k~jeho dopadení
nevedla. Nic nevíme ani o~chlapci vlevo. Vodítkem určení místa se tak
nakonec stal až rozmazaný chlapec v~dívčích šatech křepčící vedle
něho. Tím by měl být podle Třaskavého jakýsi Jára, který se prý už od
dětství choval jako debil. Podle toho, že nebyl ani na opakované výzvy
fotografa schopen chvíli postát, na tom něco bude. Jára, který si už
od mala přivydělával jako husopaska, prožil dlouhý a pestrý život.
Kromě hus pásl také ovce, kozy a několik krav. Chodil pět let do první
třídy a ani jednou nepropad. Když ho na sklonku 2. světové války během
pastvy donutila část ustupující Schörnerovy armády pod pohrůžkou
násilí aby je vzal do zajetí, nezbylo mu než se podvolit. S~medailí
kterou za to obdržel pak chodil i pod sprchu. Po únoru 1948 ho ke
vstupu do partaje nemuseli moc přemlouvat. Stal se rovněž členem
lidových milicí. Při cvičných střelbách mu ovšem pro jistotu dávali
jenom slepé náboje. Když ho po invazi v~roce 1968 dosadili pro
nedostatek vhodných kádrů do funkce okresního tajemníka strany, bylo
to nejsmutnější období jeho života. Místo svých milovaných hus, krav a
koz měl jen dvě sekretářky. A~ty on pást neuměl. I~když s~tou mladší by
to docela i šlo.

Footnote:

To, že všichni odešli současně na záchod a už se nevrátili a já zůstal
s~popsaným lístkem u~stolu sám, jsem od lidí s~tolika vysokými školami
v~tu chvíli nečekal. Na závěr jsem nucen touto cestou upozornit
historika profesora Třaskavého a jeho tým, že mi dluží 2\,403\,Kč a že
jim příště na podobnou odbornou konzultaci z~vysoka kašlu. Rovněž není
můj problém, že je v~Praze na katedře a bůh ví kde všude, blbě platěj.


