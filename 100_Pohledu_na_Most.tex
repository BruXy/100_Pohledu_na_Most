% Nastaveni jazyka

\mainlanguage[cz]
\language[cz]
\setuplanguage[cz][spacing=broad] % zapnutí větší mezery za větnou tečkou
\enableregime[utf]

% Nastaveni fontu

%\usetypescript[schola]
\usetypescript[gentium]
\setupbodyfont[gentium,10pt]

\setupbodyfontenvironment[default][em=italic]

\setupinteraction[state=start, style=normal, color=blue]

%\setupblank[flexible]

\setuppapersize[A4]

\setuplayout
    [backspace=15mm,
     width=middle,
     topspace=0mm,
     height=middle,
     bottomspace=1cm,
     footer=10mm,
     ]



% Možnosti pro vnořené uvozovky
\setupdelimitedtext[quotation:1][%
    left={\symbol[leftquotation]},
    right={\symbol[rightquotation]}%
]

\setupdelimitedtext[quotation:2][%
    left={\symbol[leftquote]},
    right={\symbol[rightquote]}%
]

% Makro \uv pro text v uvozovkách (\quotation)
\let\uv\quotation

% Odstavce
% mezery mezi odstavci
\setupwhitespace[none]

% velikost odsazeni
\setupindenting[small, next]  % none, small, medium, bix, next, first, [rozmer]
\indenting[yes]

% Iniciály

\setupinitial[n=1,continue=yes]

\setuphead[chapter][
    number=no,
    location=middle,
    style={\bf\switchtobodyfont[24pt]},
    %before={},
    after={\blank[.5em]\startcolumns[n=2]\noindentation\placeinitial},
]

\setuphead[section][number=no]

\setuppagenumbering[%
    alternative=singlesided,  % doublesided
    location={footer, center}, % header, middle, inleft, inmargin
%   location=,                % zrusi cislovani uplne
    style={\switchtobodyfont[12pt]},               %
%>··conversion=numbers,        % characters Characters romannumerals█
    left={---~},                % znak pred
    right={~---},               % znak za
%>··command=\Page,
]


% Makro pro vložení fotografie
\define[1]\foto{%
    \placefigure[top,none]{} {
       \externalfigure[#1][maxheight=0.5\textheight,maxwidth=\makeupwidth]%
    }
}

\define[1]\kapitola{
%    \startcolumns[n=2]

    \input{#1}

%    \stopcolumns
}

% Vzhled obsahu
\setupcombinedlist[content][level=1,alternative=c]
% a -- stranka hned za, b -- stranka na konci radku,
% c -- jako b, ale teckovany radek, d -- jako jeden odstavec

% Potlačení odstavcového odsazení
\let\no\noindentation

\starttext
% Debug
%\showframe
%\showsetups

\startfrontmatter
% Obsah knihy
\startcolumns[n=3]
\placecontent
\stopcolumns

\stopfrontmatter

%\input uvod.tex

% Kapitoly
\startbodymatter
\startluacode
    -- Adresář s foty nastaven v Makefile
    PHOTO_DIR = os.getenv("PHOTOS")

    -- Najdi jméno fotografie v souboru na řádku s "Foto"
    function get_image(file)
        for line in io.lines(file) do
            if line:find("Foto:") then
                -- print(line)
                -- řetězec za posledním lomítkem je lokální jméno souboru
                img_file = line:gmatch("[^/]+$")()
                if line:match(".jpg") then
                    return img_file
                else
                    -- přidej příponu pokud chybí
                    return img_file .. ".jpg"
                end
            end
        end
    end

    -- Načtení všech souborů v~adresáři vyhovující '201*.tex'
    require "lfs"

    chapters = {}   -- new array
    for file in lfs.dir(".") do
        if file:match("201.*%.tex") then
            table.insert(chapters, file)
         end
    end

    table.sort(chapters)

    for i = 1, #chapters do
        print("Including chapter: " ..  chapters[i])
        img = get_image(chapters[i])
        print("Image: " .. img)
        -- Vlož kapitolu
--        tex.print("\\startcolumns")
        tex.print(("\\input{%s}"):format(chapters[i]))
        tex.print("\\stopcolumns")

--        tex.print(("\\kapitola{%s}"):format(chapters[i]))

        -- Vlož foto
        tex.print(("\\foto{%s/%s}"):format(PHOTO_DIR, img))
    end
\stopluacode
\stopbodymatter

\input doslov.tex

\stoptext
