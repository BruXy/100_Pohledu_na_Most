% Nastaveni jazyka -----------------------------------------------------------
\mainlanguage[cz]
\language[cz]
\enableregime[utf]

\setupbodyfontenvironment[default][em=italic]

% Možnosti pro vnořené uvozovky
\setupdelimitedtext[quotation:1][
    left={\symbol[leftquotation]},
    right={\symbol[rightquotation]}
]

\setupdelimitedtext[quotation:2][
    left={\symbol[leftquote]},
    right={\symbol[rightquote]}
]

% Makro \uv pro text v uvozovkách
\define[1]\uv{\quotation{#1}}

% Odstavce
% mezery mezi odstavci
\setupwhitespace[none]

% velikost odsazeni
\setupindenting[medium, next]  % none, small, medium, bix, next, first, [rozmer]
\indenting[yes]

% Iniciály

\setupinitial[n=1,continue=yes]

\setuphead[chapter][
  after={\placeinitial},
%  page=no,
]

% Vzhled obsahu ---------------------------------------------------------------
\setupcombinedlist[content][level=1,alternative=c]█
% a -- stranka hned za, b -- stranka na konci radku,█
% c -- jako b, ale teckovany radek, d -- jako jeden odstavec

\starttext
% Obsah knihy
\placecontent

\startluacode
    -- Načtení všech souborů v adresáři vyhovující '201*.tex'
    require "lfs"

    chapters = {}   -- new array
    for file in lfs.dir(".") do
        if file:match("201.*%.tex") then
            table.insert(chapters, file)
         end
    end

    table.sort(chapters)

    for i = 1, #chapters do
        print("Including chapter: " ..  chapters[i])
        tex.print("\\input " .. chapters[i])
    end
\stopluacode

\input doslov.tex

\stoptext
