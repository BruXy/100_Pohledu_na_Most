% Zdroj: https://www.idnes.cz/zpravy/sto-pohledu/100-pohledu-na-cesko.A160721_090305_domaci_jj/diskuse
% Datum: 21. 7. 2016 10:44
% Foto:  https://1gr.cz/fotky/idnes/16/063/org/JB643b8c_V4F19.jpg

Chanov jak už ho dnes nikdo nezná. Nyní část Mostu. Řeka Bílina tudy
protéká i v současnosti, jen ty husy byste hledali marně. Obec je
známá unikátními průhlednými paneláky. Já si vzpomenu hlavně na
strejdu Eduarda. Tamního rodáka. Když jsem jednoho dne přiběhl domů,
bylo tam najednou o dost méně místa. Byla tam spousta klecí s ptáky a
nějaký chlap, kterého mi matka představila jako Edu. "Ahoj strejdo,"
uvítal jsem ho už zkušeně. Matka se na mě mile usmála. Byla ráda, že
má někoho kdo má nějakého koníčka a tudíž by nemusel jenom chlastat.
Eda nechlastal. Ale tak týden, maximálně 14 dní. Když pak za to vzal,
mohl klidně chlastat za nároďák. Ptáci řvali celý den a on se k nim
přidával večer, když zavřeli hospody. Když se vykropil, tak by pořád
něco řešil. A tak ho občas matka musela uspat lopatkou na uhlí, aby už
dal pokoj. Někdy dostal na budku už v hospodě. To ale zase pro změnu
brečel a bylo mu všechno líto. Třeba dost těžce nesl fakt, že moje
punčocha co nám připravila matka na Mikuláše, obsahovala o jeden
bonbón a dva buráky navíc. Matka problém vyřešila tím, že jsem si s
ním musel punčochu vyměnit. Při mistrovství světa v hokeji pořád
vykřikoval, že fandí těm "druhejm." Když to jednou zkusil v hospodě,
prohodili ho zavřeným oknem. Měl jsem sto chutí to při tom hokeji
udělat taky, ale ještě jsem neměl takovou sílu. Tak jsem mu aspoň
řek', že je blbej. Když se Eda dostal do situace, že mu ve všech
hospodách v okolí dali minimálně jednou přes hubu, přeřadil na
lahvový. To pak seděl doma a drezůroval ptáky. Chtěl, aby na povel
přeskočili z jednoho bidýlka na druhý a učil je mluvit. To první se mu
občas povedlo. Radost mu nekalil ani fakt, že tak činili skoro pořád.
Jednou když po něm andulky tvrdošíjně odmítaly opakovat "pepíčku", a
jen dál hlasitě vřeštěly, mrštil po nich jitrnici kterou právě
pojídal. Slušně jí nakrájel o dráty klece. Když jsem pak po čase
přišel domů, nebyl tam ani Eda, ani klece. Po klecích zůstaly hřebíky
ve zdi a po Edovi mastnej flek tamtéž.

