% Zdroj: https://www.idnes.cz/zpravy/sto-pohledu/100-pohledu-na-cesko.A160512_131155_domaci_hro/diskuse/2
% Datum: 13. 6. 2016 11:41
% Foto: https://1gr.cz/fotky/idnes/16/062/org/JAV63eb5c_fotofoto.jpg

Vidíme účastníky lovu na řece Bílině, protékající Mostem. Někdy ke
konci 19. století. Vraťme se ale ještě hlouběji do historie této
významné řeky, a to k problematice říčního pirátství, které zejména
počátkem 17. století, v této oblasti doslova kvetlo. K nejaktivnějším
v tomto oboru podnikání, patřila tlupa vedená loupeživým rytířem
Lájošem mladším. To byl protřelý říční pirát, jehož dřevěná noha
dávala tušit, že už má v této branži leccos za sebou. O část levé
nohy, přišel během pobytu v Jižní Americe, kam ho vyslal sbírat
zkušenosti jeho otec, Lájoš z Třebušic u Mostu, jehož rod vlastnil už
od nepaměti jednu z parcel v tamní zahrádkářské kolonii. Lájoš junior
měl smůlu. Když se mu noha ve stavu mírné opilosti svezla mimo člun do
Amazonky, v místech hojně obývaném piraněmi, moc mu z ní nezbylo. O to
více si ale přivezl zkušeností. On a jeho lidé, byli v přepadávání
plavidel na řece Bílině mimořádně úspěšní, protože dokázali pružně
měnit taktiku. Jejich rychlé loďky, neměly pražádný problém
dostihnout nemotorné pramice a těžkopádné vory obchodníků naložené
vším možným. Dost často také své oběti přepadávali ze zálohy, ukryti
za rohem řeky. Neštítili se dokonce ke svým zločineckým rejdům využít
i zvířata. Konkrétně bobry. Z jejich pomocí dokázali přehradit řeku a
samotné loupení už bylo hračkou. Bezohlední piráti, pak nejenom, že
nebohé kupce okradli ale naházeli je i do řeky, a ti, protože až na
vyjímky neuměli plavat, byli nuceni opustit řeku po svých. Konec
zlatým časům pro Lájoše a jeho bandu nastal s příchodem třicetileté
války. A když pak krajem táhly švédské hordy, které kradly, co jim
přišlo pod ruku, došlo i na Lájoše. Stárnoucí pirát přišel úplně o
všechno, co si za celá ta léta pracně nakradl. Dokonce i o svou
dřevěnou nohu. Na sklonku života tak musel pracovat. Nejprve jako
údržbář a později jako noční hlídač ve skladu svíček, v klášteře
magdalenitek. V Mostě~-- Zahražanech.
