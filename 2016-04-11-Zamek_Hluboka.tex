% Zdroj: https://www.idnes.cz/zpravy/sto-pohledu/100-pohledu-na-cesko.A160411_092921_domaci_jj/diskuse/3
% Datum: 11. 4. 2016 10:41
% Foto:% https://www.idnes.cz/rF1J1Sb3zusgO55TeXmAhJAIO4NeUIc3r5MkCO0BnyyjMXp0uTVmMKYrDbQL/9tOIAH99TXEXxZbMWR5NNlI5OjQK8bvjlwjTGYGTfHxuWmE

Most, perla Sudet. To drobné, tam mezi stromy, jsou japonští turisté.
V~té době bylo město největším tahákem cestovních kanceláří. Zájem o
tuto destinaci ještě vygradoval, když uhlobaron Schwartzschweinn
přišel s nápadem, kdy by si zhýčkaní turisté mohli na vlastní kůži
vyzkoušet těžbu uhlí. Samozřejmě za tučný příplatek. Ale i tak si
památkami a muzei znudění turisté, nemohli tento druh adrenalinové
dovolené vynachválit. Občas se stalo, že si v případě závalu, či
výbuchu museli svou dovolenou v dole prodloužit. Takovým se pak
dovolená znatelně prodražila. Prý jsou i tací, kteří jsou v podzemí
dodnes... Majitelům dolů se tento druh podnikání samozřejmě zamlouval,
nepotřebovali totiž prakticky žádné vlastní zaměstnance a tak když si
připočteme zisky z takto vytěženého uhlí, peníze se jenom hrnuly.
Bohužel ve snaze zviditelnit se, využili této situace někteří jedinci
a štvali propuštěné horníky proti svým bývalým zaměstnavatelům.
Obzvláště aktivní byl v tomto směru nějaký Gottwald a jeho parta(j).

