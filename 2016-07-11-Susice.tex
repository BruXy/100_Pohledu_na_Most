% Zdroj: https://www.idnes.cz/zpravy/sto-pohledu/100-pohledu-na-cesko.A160711_100443_domaci_jj/diskuse
% Datum: 11. 7. 2016 11:47
% Foto: https://1gr.cz/fotky/idnes/16/063/org/JB641419_V4D12u.jpg
% Body: +114 -0

\chapter{Most jako na dlani}

Na archivním snímku vidíme dva z~mosteckých kostelů, na které se při stěhování
nedostalo. Vpravo nahoře je tamní muzeum. Lví podíl na jeho vzniku měl starosta
von Pohnert, který roku 1888 oslovil občany města, aby darovali vhodné
předměty. Moc se jich nesešlo. Úspěch se dostavil, až když sháněním exponátů
pověřil mosteckého výběrčího daní, který měl volný přístup do všech domácností.

V~muzeu pracovala dlouhá léta i paní Drclá. Nejprve vedla expozici novověku a
když byla starší, byla přeřazena mezi vykopávky. Jednou nás s~Leošem za odměnu
pozvala do muzea na prohlídku, protože jsme zachránili pana Drclého před
utopením v~louži, ve které usnul při návratu z~hostince. Ten se nám celou
cestu, co jsme ho vlekli domů chlubil, že právě jako první Čech zdolal kanál La
Manche stylem ouško.

Paní Drclá nás do muzea vzala po zavírací hodině a říkala, že nás tak aspoň
nebude nikdo při prohlídce rušit. Leoš zařídil, že nás nerušila ani paní Drclá,
protože jí zamknul na záchodě. Ředitel muzea se se svou mladou sekretářkou
zamkli pro jistotu taky. Z~kanceláře se pak ozývaly zvuky, jako by tam
stěhovali nábytek. V~muzeu to bylo moc fajn, jen nám bylo divné, že v~oddělení
pravěku, byl jeden z~lovců mamutů oblečen v~montérkách. Když jsem do něj
opatrně zapíchnul oštěp, podíval se na hodinky a když zjistil, že už pracuje
půl hodiny přes čas, odebral se k~píchačkám.

Nejvíc se nám líbilo v~oddělení bodných a sečných zbraní. Akorát tam trochu
táhlo, protože se mi podařilo vysklít dvě okna při manipulaci s~halapartnou.
Když nás přestalo bavit se šermovat a všichni ti rytíři v~brnění leželi na
zemi, přesunuli jsme se do oddělení palných zbraní. Tam jsme ale dlouho
nepobyli, protože nás vyrušila paní Drclá ve chvíli když jsme se pokoušeli
vystřelit z~děla. Tu ze záchoda vysvobodila sekretářka, která už měla stejně
jako ten pračlověk v~montérkách ten den padla. Paní Drclá uhasila doutnák,
mluvila sprostě a hnala nás z~muzea. Nakonec sebrala Leošovi i ty dvě bambitky,
co si vzal na památku.

