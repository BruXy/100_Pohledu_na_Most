% Zdroj: https://www.idnes.cz/zpravy/sto-pohledu/sto-pohledu-na-cesko-uhercice.A161003_092937_domaci_hro/diskuse/3
% Datum: 3. 10. 2016 10:41
% Foto: https://1gr.cz/fotky/idnes/16/101/org/JB66612a_ha.jpg
% Body: +225 -0

\chapter{Bečov u~Mostu}

Odtud pocházel můj kamarád Míra. Basák se kterým jsme založili rockovou kapelu
{\em Padající trakaře}, protože jsme chtěli snáz balit holky. Míra mi uměl
naladit i kytaru, protože hrál jako malej na tahací harmoniku. Na bicí s~náma
tehdy hrál Bob z~Bylan, který už měl zkušenosti z~tamní kapely {\em Led
Sezelím}. Ovšem založit za totáče kapelu byla jedna věc a mít možnost hrát
alespoň na vesnických tancovačkách věc druhá.

Aby nám to povolili, museli jsme absolvovat tzv. přehrávky. Ty naše se tenkrát
konaly v~mosteckém Neprakta klubu. Bylo tam ten den asi pět dalších kapel. Těch
soudruhů a soudružek v~té komisi asi taky tolik. Šli jsme hned jako první. Měli
jsme zahrát pět skladeb s~převahou českých textů. S~tím jsem neměl problém,
skládat mi šlo celkem dobře. Začali jsme středně rychlou \uv{Tys jí včera
brachu, měl zase jak z~praku}, jejíž refrén \uv{Čtyři páry bílejch telat, mně
se dneska nechce dělat, na práci se vyseru, radši se tu vožeru}, jsem opsal na
záchodě v~nádražní restauraci v~Chomutově. Na tom záchodě toho tenkrát bylo od
neznámého autora trochu víc. Vzpomínám si ještě na \uv{Čtyři páry bílejch myšek,
nachcaly mi do semišek, jsou to myšky potvory, lížou sobě otvory}.

Vypadalo to nadějně, členové komise zůstali civět s~otevřenými ústy. Než je
stačili zavřít, dali jsme song \uv{Vášnivá soudružka}. Bohužel se jednomu
z~členů komise patrně nezdálo něco na jejím refrénu znějícím \uv{Pod portrétem
Husáka, sahala mi na ptáka} a vytrhl napájení aparatury ze zdi i se zásuvkami,
čímž způsobil zkrat, který dost možná shodil hlavní jistič až v~hydroelektrárně
v~Kujbyševě. Pak na nás začal řvát, že ať koukáme okamžitě vypadnout, nebo nás
všechny pozavíraj i s~nima. Na skladby \uv{Přijel jsi k~nám drahou, chtěls mě
vidět nahou}, \uv{Měl jsem jí jednou, za švédskou bednou} se tak už nedostalo,
stejně jako na feministicky laděnou \uv{Nebudu ti prát, když ti nechce stát}.
Po kapele Padající trakaře tak nakonec zůstalo jedno jediné album, které jsme
natočili\footnote{Pásek s originálním záznamem se bohužel nedochoval. Údajně na
něj Mírův otec natočil nějakou dechovku.} u~Míry v~pokoji na magnetofon B-100.

Až po letech jsem pochopil, že svět ještě nebyl na nástup punku připraven a že
kapely jako Sex Pistols, které se jím nakonec proslavily, byly vlastně jen
takový náš revival.
