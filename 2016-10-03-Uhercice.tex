% Zdroj: https://www.idnes.cz/zpravy/sto-pohledu/sto-pohledu-na-cesko-uhercice.A161003_092937_domaci_hro/diskuse/3
% Datum: 3. 10. 2016 10:41
% Foto: https://1gr.cz/fotky/idnes/16/101/org/JB66612a_ha.jpg

Bečov u Mostu. Odtud pocházel můj kamarád Míra. Basák se kterým jsme
založili rockovou kapelu "Padající trakaře", protože jsme chtěli snáz
balit holky. Míra mi uměl naladit i kytaru, protože hrál jako malej na
tahací harmoniku. Na bicí s náma tehdy hrál Bob z Bylan, který už měl
zkušenosti z tamní kapely Led Sezelím. Ovšem založit za totáče kapelu
byla jedna věc a mít možnost hrát alespoň na vesnických tancovačkách
věc druhá. Aby nám to povolili, museli jsme absolvovat tzv. přehrávky.
Ty naše se tenkrát konaly v mosteckém Neprakta klubu. Bylo tam ten den
asi pět dalších kapel. Těch soudruhů a soudružek v té komisi asi taky
tolik. Šli jsme hned jako první. Měli jsme zahrát pět skladeb s
převahou českých textů. S tím jsem neměl problém, skládat mi šlo
celkem dobře. Začali jsme středně rychlou "Tys jí včera brachu, měl
zase jak z praku", jejíž refrén "Čtyři páry bílejch telat, mně se
dneska nechce dělat, na práci se vys..u, radši se tu vožeru", jsem
opsal na záchodě v nádražní restauraci v Chomutově. Vypadalo to
nadějně, členové komise zůstali civět s otevřenými ústy. Než je
stačili zavřít, dali jsme song "Vášnivá soudružka". Bohužel se jednomu
z členů komise patrně nezdálo něco na jejím refrénu znějícím "Pod
portrétem Husáka, sahala mi na ptá.a" a vytrhl napájení aparatury ze
zdi i se zasuvkami, čímž způsobil zkrat, který dost možná shodil
hlavní jistič až v hydroelektrárně v Kujbyševě. Pak na nás začal řvát,
že ať koukáme okamžitě vypadnout, nebo nás všechny pozavíraj i s
nima. Na skladby "Přijel jsi k nám drahou, chtěls mě vidět nahou",
"Měl jsem jí jednou, za švédskou bednou" se tak už nedostalo, stejně
jako na feministicky laděnou "Nebudu ti prát, když ti nechce stát". Po
kapele "Padající trakaře" tak nakonec zůstalo jedno jediné album,
které jsme natočili u Míry v pokoji na magnetofon B-100. Až po letech
jsem pochopil, že svět ještě nebyl na nástup punku připraven a že
kapely jako Sex Pistols, které se jím nakonec proslavily, byly vlastně
jen takový náš revival.
