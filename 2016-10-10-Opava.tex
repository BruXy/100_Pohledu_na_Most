% Zdroj: https://www.idnes.cz/zpravy/sto-pohledu/100-pohledu-na-cesko.A161010_083524_domaci_jj/diskuse/2
% Datum: 10. 10. 2016 10:36
% Foto: https://1gr.cz/fotky/idnes/16/101/org/JB666286_aa.jpg
% Body: +107 -1

Tržní náměstí ve starém Mostě. Ten muž vpravo, který může na
procházející ženě nechat oči, je Ferdinand Vošoust. Bohém a básník
známější pod uměleckým jménem Fráňa Trámek. Až donedávna se mělo za
to, že celé jeho dílo je nenávratně ztraceno. Před časem se však
podařilo profesoru Úlisnému objevit v jednom opuštěném skladu desky z
lavic, ve kterých sedával mladý Vošoust při svém studiu na reálném
gymnáziu v Mostě a do kterých vyryl část své rané tvorby. A tak i přes
pozdější necitlivé nátěry lavic, se panu profesorovi podařilo část
díla tohoto mosteckého velikána zachránit. Pro ty, kteří se s dílem
Fráňi Trámka dosud nesetkali, uvádím několik ukázek z lavice ve které
sedával jako žák 3.B.

Nechala mě Vladana,
hrát s jejíma vnadama,
když jsem se šel vysprchovat,
zmizela mi s prachama.

Když stála ve tmě,
zdála se hezká.
Jak točila kabelkou,
vidím jak dneska.

Že prý bych mohl být,
pro dnešek první.
Mé druhé já velelo,
nebuď vůl, zdrhni.

Tomu dala, tomu taky,
jenom mně dát nechtěla.
Tak jsem jí řek že si najdu,
jinde otvor do těla.

Našel jsem jí pod peřinou,
byla ještě vlahá,
A ten chlap v tom prádelníku,
byl náš soused Bláha.
