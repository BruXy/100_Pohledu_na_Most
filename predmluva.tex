\define\vlasta{%
    \placefigure[right]{Vlasta Beneš, autor}{%
       \externalfigure[vlasta_benes_ai.jpg][maxwidth=0.35\makeupwidth]
    }
}

\chapter{Předmluva}

Historické severočeské město Most bylo v~70. a 80. letech 20. století téměř
zcela zbořeno kvůli těžbě uhlí. Nejcennější stavbou a turistickým lákadlem je
kostel Nanebevzetí P. Marie, roku 1975 přesunutý na současné místo, a hrad
Hněvín na kopci nad městem.

Sbírka historických fotografií původního Mostu v~okresním archivu čekala řadu
let na přesné uřčení a doplnění. Byl přizván znalec {\em Vlasta Beneš}, který
se zhostil tohoto nelehkého úkolu.

\section{O~autorovi}

Vlasta Beneš (*\,1953) se narodil ve městě Most a prožil zde i celý
svůj život. Z~Mostu a okolí pochází i jeho rodiče a prarodiče.


Již v~mládí se ukázalo, že je všestranný talent, jeho zájmem se stal nejen
sport, ale i umění. Učil se hrát na nejrůznější hudební nástroje, tělo tužil
při kopané a boxu. S~přáteli založil bigbeatovou kapelu, která ovšem nedostala
oficiální povolení k~produkci.

S~úžasem pozoroval zkázu historického města, které ustupovalo důlním strojům a
výstavbu města zcela nového. To v~něm zanechalo touhu po zachování historie
města, jeho obyvatel, vztahů a příběhů. Tak jak identifikoval místa na
fotografiích, doplnil je vlastními vzpomínkami s~nimi spojenými.

Další z~výjimečných zásluh je příspěvek při znovuobjevování mosteckého básníka~--
{\em Fráňa Trámek} (*\,1884~-- \dag{}\,1911), vlastním jménem Ferdinand Vošoust,
vytvořil za svůj krátký život několik básnických sbírek. Historici byli o~jeho
díle přesvědčeni, že je nenávratně ztraceno. Ale část se podařila zachránit a
díky Vlastovy přinést a zpopularizovat i mezi širší veřejností.

Fráňa Trámek zemřel v~mladém věku 27 let na komplikace spojené s~gumatózním
stádiem příjice, které si neúspěšně léčil v~mosteckých lázních.

\vlasta

\stopcolumns % uzavření za sloupců za chapter/after

