% Zdroj: https://www.idnes.cz/zpravy/sto-pohledu/100-pohledu-na-cesko.A161121_091248_domaci_jj/diskuse/2
% Datum: 21. 11. 2016 10:40
% Foto: https://1gr.cz/fotky/idnes/16/112/org/JB673875_inyH2107.jpg
% Body: +129 -3

\chapter{Májové slavnosti l.\,p.\,1905}

Májové slavnosti v~Mostě někdy kolem roku 1905. Zcela nahoře na jejich
hladký průběh dohlíží dva uniformovaní četníci. Těm se kvůli chocholům
na přilbách říkávalo {\em chlupatý}. Ten s~tím mohutnějším knírem je
wachtmeister Kloutschek. Toho druhého neznám. Trochu doleva po nimi
stojí dva policajti v~civilu. O~těch se mi podařilo zjistit pouze to,
že ten menší a tlustší měl na sobě v~té době světlejší oblek. Močící
pán u~sloupku se ohleduplně otočil k~dětem zády. Pravé ruce třímá
džbánek s~pivem, co drží v~té levé, není až tak těžké uhodnout. Pak tu
máme dva muže stojící před ním. Ten co má postoj, klobouk a plášť
jako Napoleon Bonaparte, se za něho opravdu považoval a byl na
procházce se svým ošetřovatelem. Tomu dalo hodně práce mu vysvětlit,
že to co právě vidí není bitva u~Slavkova. Tématem mužů kráčejících
dolů po chodníku, nebudou zajisté pouze kotníčky a pozadí dam jdoucích
kousek před nimi. V~domě číslo 47, ve třetím patře v~bytě č.\,13, nebyl
v~tu dobu nikdo doma. Dívenky v~bílém, které utvořily kolem májky
neprodyšný kruh, jsou ze sudetoněmeckého souboru písní a tanců Küken
von der Brüx. Uprostřed kruhu lze vidět jeho ředitele Řehoře
Kropatscheka a dvě ženy. Ta vpravo s~většími prsy, byla tou dobou jeho
manželkou. Ta nalevo, s~nižším číslem podprsenky a počtem svíček na
narozeninovém dortu, jí nejprve připravila o~místo umělecké vedoucí
souboru a krátce na to i o~Kropatscheka. Skupince dětí na stráni vlevo
se při neopatrné manipulaci s~rýčem podařilo smrtelně zranit krtka.
Kněz který k~nim spěchá zleva, má zřejmě v~úmyslu poskytnout mu
poslední pomazání. Hvězdami oslav toho roku byly tanečnice
z~folklórního souboru Pozsony Táncbanda, které vidíme na chodníku
uprostřed snímku. Na činnost tohoto seskupení navázal o~několik
desítek let později soubor Lúčnica. Slovanský šarm dívek oslnil dva
hochy oděné jako pážata natolik, že odhodili prapory a jen obdivně
zírali. A~to ještě netuší, že další nádhera se k~nim blíží z~pravého
dolního rohu v~podobě členek Sokola Most.

