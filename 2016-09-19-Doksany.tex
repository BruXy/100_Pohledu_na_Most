% Zdroj: https://www.idnes.cz/zpravy/sto-pohledu/100-pohledu-na-cesko-doksany.A160919_101430_domaci_hro/diskuse/2
% Datum: 19. 9. 2016 11:44
% Foto: https://1gr.cz/fotky/idnes/16/092/org/JB66125e_Dy57.jpg

V tomhle mosteckém zámku sídlila v dobách mého dětství redakce
okresního týdeníku Rozvoj. Jednou k nám do třídy přivedl soudruh
ředitel nějakého tlustého chlápka a řekl, že je to soudruh Líný z
Rozvoje a bude psát článek o naší škole. Pak ještě dodal, že doufá že
neuděláme škole ostudu a odešel něco důležitého řešit se svou mladou
sekretářkou. Líný si sedl na místo naší soudružky učitelky a začal se
nás jednoho po druhém vyptávat na různé otázky. Když vyzpovídal asi
polovinu třídy, vypadal docela naštvaně. Zatímco jeho obličej víc a
víc brunátněl, ten soudružky učitelky, opírající se u okna o parapet,
stejnou rychlostí bledl. Popravdě ani nebylo divu. Dozvědět se
najednou tolik závažných informací, to s jedním zacloumá. Například
to, že prvním člověkem ve vesmíru byl Lenin, Klement Gottwald byl
spoluzakladatelem automobilky Laurin & Klement, a že Brežněv je
hokejista. A docela dobrej. Na dotaz ohledně Slovenského národního
povstání, mu Čenda odpověděl, že sice neví nic konkrétního, ale určitě
se něco chystá. Když mu pak Gejza, co sedával v poslední lavici
sdělil, že se na jméno prvního tajemníka KSČ zeptá doma táty, protože
ten s ním seděl na Borech, soudružka učitelka zřejmě z taktických
důvodů omdlela. Okamžitě nás několik vyběhlo pro pomoc. Naštěstí
školník zrovna vytíral chodbu. Když na soudružku učitelku vyždímal
hadr, docela rychle se probrala. Školník byl vůbec ten den za hrdinu.
Nejen že zachránil naší soudružce život, ale ještě se mu podařilo
odlákat redaktora k sobě do kamrlíku na domácí slivovičku. Když pak v
průběhu další hodiny Líný vevrávoral do třídy, byla s ním docela
legrace. Kravatu měl jako indián čelenku a z poklopce mu koukala
košile. Blekotal jak se mu u nás líbilo a když se loučil se soudružkou
učitelkou, tak jí sahal na zadek a ptal se jí, co dělá večer. Ta se
poprvé ten den začervenala. Článek, který v následujícím týdnu vyšel v
Rozvoji, byl oslavou žáků i zaměstnanců školy. A kdyby tu školu
nezbourali, visel by tam na nástěnce ještě dneska.
