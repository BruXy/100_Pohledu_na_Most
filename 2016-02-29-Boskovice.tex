% Zdroj: https://www.idnes.cz/zpravy/sto-pohledu/100-pohledu-na-cesko.A160229_090242_domaci_jav/diskuse/3
% Datum: 29. 2. 2016 9:37
% Foto: https://www.idnes.cz/sH1J1Sb3zusgO55TeXmAhJAIO4NeUIc3r5MkCO0Bnx6OJhZDsXuki6or3fbm/rousYGE0IdwoPQCRkBjwcsOiSPJFqu0U8YpW7HixmDVaOnG

Tak je to pro změnu opět Most. V domě na pravé straně, u kterého je
přistaveno lešení, býval vyhlášený hotel Grand. Traduje se, že zde
nocoval císař Napoleon při svém tažení na ostrov Svatá Helena. Vše je
však zahaleno rouškou tajemství. Část historiků argumentuje tím, že
chybným překladem ze starofrancouzštiny, přes tehdy hojně využívanou
staroněmčinu do staročeštiny, došlo k mylnému výkladu pojmů. Např.
profesor Dbalý sice připouští existenci jakési Heleny, ale ta prý
rozhodně svatá nebyla a údajně se živila jako společnice VIP hostů.
Což v té době císař bezesporu byl. Jak to tedy bylo doopravdy s
hotelem Grand, Napoleonem a (Sv.) Helenou se tedy už dnes
pravděpodobně nedozvíme. Obzvláště, když knihu hostů hrubě poničili
nějací neukáznění švédští turisté během třicetileté války.

