% Zdroj: https://www.idnes.cz/zpravy/sto-pohledu/100-pohledu-na-cesko.A160825_090545_domaci_jj/diskuse/3
% Datum: 25. 8. 2016 10:16
% Foto:

Kamený most v Mostě. Podle nízkého stavu vody v řece Bílině, se
pravděpodobně jedná o suchý rok 1904. Ten uličník clonící si oči bude
nejspíš pan Poděs. Budoucí okresní přeborník v bězích na středních
tratích. "Pošlete toho vašeho kluka někdy za mnou mladá pani," povídá
jednou v mlíkárně mé matce pan Poděs, který už v té době vedl pouze
atletický oddíl u nich v domově důchodců. "Když jsem ho onehdá viděl
běžet a skákat přes ploty, když upaloval s těma kradenejma hruškama od
Nepřejícných, docela mu to šlo. Takové talenty je třeba včas
podchytit, než nám je odchytí SNB a následně předá vězeňské službě. To
pak ale bývá už většinou pozdě." Na to matka slyšela. Nepřála si abych
skončil v kriminále, jak mi opakovaně prorokovala paní Untrmanová od
nás z přízemí. A tak jsem se druhý den, oděn do sportovního, hlásil u
pana Poděsa. Ten byl rád, že má konečně někoho perspektivního, protože
jak říkal, jeho současní svěřenci jsou už několik desítek let za
zenitem. Taky povídal, že jsou líný jak prasata a pořád se na něco
vymlouvaj. Hlavně ti s berlema. I oni mě rádi viděli, protože doufali,
že budou mít o něj konečně pokoj. Hody a vrhy mi moc nešly. Kladivo
jsem ani neuzdvihnul, kouli jsem si málem hodil na nohu a oštěpem jsem
propíchnul trenérovi novou bundu, co si odložil vedle na lavičku. Ta
děravá bunda ho asi dost mrzela, protože když mi vyklouznul disk a já
ho s ním trefil pod koleno, tak mi ho sebral a ještě mi dal facku.
Skákání bylo už lepší. Jen u výšky mi vadila ta laťka, ale stejně jí
tam pořád dával. U skoku do dálky jsem měl zase pocit, že ten písek
vysypali zbytečně moc daleko. Nejvíc si ale pan Poděs sliboval od mého
běhání. Vždy si přál vychovat svého nástupce. Mě to běhání ale
nebavilo a jeho asi taky ne, protože ne mě pořád řval a ještě k tomu
sprostě. Hlavně jsem nechápal, proč bych měl rychle utíkat, když jsem
nic neproved. Tvář se mu rozjasnila, až když na mě poštval svýho psa.
Tomu psovi jsem tenkrát utek, ale k tomu poděsovi už mě nikdo
nedostal.

