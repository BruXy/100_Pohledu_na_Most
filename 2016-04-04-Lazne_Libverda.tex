% Zdroj: https://www.idnes.cz/zpravy/sto-pohledu/100-pohledu-na-cesko.A160401_125357_domaci_jav/diskuse/2
% Datum: 4. 4. 2016 10:30
% Foto: https://1gr.cz/fotky/idnes/16/041/org/JB625152_00.jpg
% Body: +66 0

\chapter{Okrajová část Souše}

Krásný dobový snímek okrajové části Mostu. Souše, lidově Čauše. V~pozadí vidíme
panoramata Krušných hor. Ty, na rozdíl od starého Mostu, kupodivu stojí dodnes.

Na tom kopečku před nimi, stával s~různými přestávkami hrad Hněvín.
Snímek byl tedy pořízen někdy před rokem 1905, kdy byl postaven ten
současný. Ten úplně původní, byl zbořen samotnými mostečany, když se
dozvěděli, že se jej chystají dobýt husité pod vedením Jana Žižky. Ti
nějaký čas zmateně pobíhali mezi kopci Českého středohoří, marně
hledaje hrad. Když se navíc lstivým mostečanům podařilo mezi ně
rozšířit zprávu, že všechny ženy v~okolí už byly znásilněny, ztratili
o~tuto oblast zájem a odtáhli dobývat, plenit a znásilňovat neznámo
kam.
