% Zdroj: https://www.idnes.cz/zpravy/sto-pohledu/100-pohledu-na-cesko.A160502_085842_domaci_jj/diskuse/3
% Datum: 2. 5. 2016 10:09
% Foto: https://1gr.cz/fotky/idnes/16/041/org/JB6251b8_X1204.jpg
% Body: +101 0

\chapter{Přístav na řece Bílině}

Na snímku je přístav na řece Bílině, lidově Běle a část mostecké
průmyslové zóny. Na tom kopci v~pozadí, vidíme část bývalé továrny na
výrobu parních počítačů. Tam pracoval nějaký čas i můj děda, když
babička zjistila, že sama všechny neuživí. Ve výrobě těchto počítačů,
jsme v~té době patřili k~absolutní světové špičce a tyto byly vyváženy
do celého světa. Jejich nevýhodou ale byla vyšší cena a tak si je
mohli dovolit jen velké firmy a zámožní klienti. Dalšími zápory
pak byla velká hmotnost, hlučnost, negativní vliv na životní
prostředí a v~neposlední řadě i nezanedbatelné provozní náklady.
Na výpočet příkladu z~malé násobilky, byl potřeba jeden pytel
tříděného uhlí. Na vyřešení rovnice o~jedné neznámé, pak pytle
tři. Navíc bylo třeba kromě samotné obsluhy počítače, zaměstnávat
ještě topiče. A~tak když se Japoncům a Američanům podařilo
vyvinout levnější a dokonce výkonnější výpočetní techniku, byla
výroba parních počítačů v~této továrně postupně utlumena. Naštěstí
pro tento region, se díky stávajícímu technologickému vybavení
továrny, mohlo plynule přejít na výrobu důlních strojů a
velkostrojů. Ty navíc nebylo nutno nikam daleko přepravovat,
protože se mohli zakousnout do krajiny hned kousek za městem a
nakonec i do něho samotného.
