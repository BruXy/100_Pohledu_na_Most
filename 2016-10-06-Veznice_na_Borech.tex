% Zdroj: https://www.idnes.cz/zpravy/sto-pohledu/100-pohledu-na-cesko-plzen-bory.A161006_092705_domaci_hro/diskuse/3
% Datum: 6. 10. 2016 11:02
% Foto: https://1gr.cz/fotky/idnes/16/101/org/JB6661c3_rr.jpg
% Body: +157 -0

Silnice vedoucí do Mostu. Tudy se koncem srpna 1968 valila vojenská
technika spřátelených armád. Dělo se toho tenkrát hodně. Z našeho
soudruha učitele byl pan profesor, protože než ho vylili k nám na
základku, vyučoval v Praze na univerzitě. Časopis Pionýr se začal
jmenovat Větrník a Karlíkův otec se naštval na "rusáky" tak moc, že
odhlásil Svět sovětů. A pak do naší třídy přišel pan Dušín, co jezdil
na dole Ležáky s fungl novou skříňovou Avií, že zakládá skautský
oddíl. To se mi líbilo. Do "pionýra" mě nikdy nevzali a tady se mě
nikdo na prospěch a známku z chování neptal. Vyfasoval jsem takové
khaki montérky a červený baret v kterém jsem vypadal jako debil. Tak
jsem ho hned zkraje ztratil. Netrvalo dlouho a pan Dušín nás naskládal
do Avie a my vyjeli tábořit, jak se na správné skauty sluší. Na louce
    obklopené lesem jsme postavili stany, vykopali latrínu a nasbírali
    dřevo na táborák. Pan Dušín nám opakovaně vštěpoval heslo "Jeden
    za všechny, všichni za jednoho". A tak když tlouštík Ivo za nás za
    všechny sežral do jedné námi nasbírané borůvky, co měly být na
    palačinky, dali jsme mu teda všichni svorně do huby. Sice se
    snažil zapírat, ale tak fialová huba se fakt nevidí. Také jsme
    hráli různé hry. Při jedné z nich propadnul Alfréd tím kulatým
    otvorem v kadibudce. Protože už dost smrděl, tak mu nikdo z nás
    nechtěl podat ruku a pan Dušín prohlásil, že si kvůli tomu volovi
    nenechá za.rat novej provaz. Tak jsme tu budku nakonec povalili.
    Došlo i na dobré skutky. Například jsme zachránili hajnému
    manželství, když jsme mu neřekli, že pokaždé jak odjede na polesí,
    zahýbá mu stará s některým z lesních dělníků. Taky jsme vrátili
    před hospodu to kolo. Stejně už bylo píchlý. Škoda, že tam nebyly
    žádný holky. K čemu jsou ty hrdinské činy a dobré skutky, když to
    žádná z nich neocení aspoň pusou. Ale tohle skautské období trvalo
    jen pár měsíců. Než jsme se nadáli, byl z pana profesora zase
    soudruh učitel, z Větrníku Pionýr a pan Dušín neřídil ani tu Avii,
    natož skautský oddíl.

