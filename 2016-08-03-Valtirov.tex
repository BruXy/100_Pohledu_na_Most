% Zdroj: https://www.idnes.cz/zpravy/sto-pohledu/sto-pohledu-na-cesko-valtirov.A160802_105906_domaci_hro/diskuse
% Datum: 3. 8. 2016 14:05
% Foto:  https://1gr.cz/fotky/idnes/16/073/org/JB64c79f_chotek.jpg

Tuto část Mostu neznám. Patrně byla odtěžena, ještě než jsem se
narodil. Dlouho jsem ve svém archívu hledal nějaký záchytný bod a tím
se nakonec stala dívka na snímku. Nejspíš to bude paní Untrmanová, co
bydlela u~nás v~přízemí. Za mlada, samozřejmě. Možná se kaje za své
budoucí hříchy. Paní Untrmanová věděla o~každém v~našem okolí úplně
všechno. I~to, co nebyla pravda. Když ještě žil pan Untrman a
v~hospodě na růžku se blížila zavírací hodina, bylo mu ho všem upřímně
líto. Nás děti neměla paní Untrmanová ráda. Nadávala, že moc řveme a
ona pak neslyší o~čem se baví dospělí. Když ale potřebovala pro něco
dojít do obchodu, dala by se namazat na chleba. Když se mě takhle
jednou opět pokoušela odchytit, řekl jsem, že se musím jít domů učit,
protože chci mít na konci roku samé jedničky a opravit si tak tu jednu
dvojku z~pololetí. Začala ječet, že moc dobře ví, že jsem samá trojka
a že stejně skončím u~lopaty. Taky říkala, že ta dvojka z~chování je
pro mě ještě dost málo. Nakonec dodala, že jsem celej můj fotr. Toho
taky vůbec nemusela. Jednou jí při návratu z~hostince řekl, že kdyby
jí v~tom okně někdo vyfotil, byla by to pěkná etiketa na vepřovou
konzervu. Když si pak na jeho počínání ztěžovala matce, tak ta se jí
snažila uklidňovat tím, že to určitě nemohl myslet vážně, poněvadž by
ty konzervy nechtěl nikdo ani zadarmo. Za nějaký čas došlo při
přesídlování do paneláků i na naši ulici. Paní Untrmanovou
přestěhovali do domu, kde se s~nikým neznala a tudíž neměla žádný
respekt. Jen jednou si zkusila poslat některé z~mnoha sousedovic dítek
pro něco do obchodu. Neviděla ani nákup, ani peníze. Ani následná
reklamace u~rodičů dítěte, nebyl dobrý nápad. Když přišla po těch
penězích ještě o~zubní náhradu, došlo jí, že jakákoli další konverzace
je zbytečná. Po mnoha dalších peripetiích v~novém bydlišti usoudila,
že si snáze sjedná respekt v~domově důchodců. Tam dost možná žije
dodnes. Panu Untrmanovi bych ještě pár let klidu na hřbitově docela
přál.

