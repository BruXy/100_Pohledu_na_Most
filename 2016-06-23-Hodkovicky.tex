% Zdroj: https://www.idnes.cz/zpravy/sto-pohledu/100-pohledu-na-cesko.A160623_092929_domaci_jav/diskuse/2
% Datum: 23. 6. 2016 11:43
% Foto: https://1gr.cz/fotky/idnes/16/063/org/JB641407_M1102taa.jpg

Tak toto je opravdu vzácný snímek pořízený na břehu řeky Bíliny.
Odhaduji to tak na rok 1905, protože na kopci za řekou, vidíme ne
zcela dostavěný hrad Hněvín. Samozřejmě že to krásné město, ze kterého
nevidíme tentokrát ani kousek je Most. Ovšem mnohem zajímavější než
samotné město, jsou tentokrát postavy zachycené na této unikátní
fotografii. Začněme těmi čtyřmi pány zcela vlevo, dle oblečení
patřícími k vyšší střední třídě. Ti zcela evidentně čekají na takzvaný
"koňský přívoz", který je dopraví na druhý břeh. Převozníka s koňmi
vidíme v pozadí. Důkazem, že si tento způsob přepravy mohli dovolit
jen ti movitější, je matka samoživitelka, která se svými třemi dětmi
zdolává řeku pěšky. To je ta postava a tři hlavy uprostřed řeky. A
teď k tomu nejpodstatnějšímu. Ta půvabná dáma je totiž jedna z
mořských panen, které se do této oblasti připlouvaly z Hamburku po
Labi každé léto třít. Že prožívá plodné dny, je patrno z jejího
spokojeného výrazu. Usmívá se i přes to, že je právě ošetřována
místním lékařem, když si zřejmě v mělké vodě poranila ploutev. Ten
usmívající se pán v plavkách, je z největší pravděpodobností její
zachránce. Že byl o páření se s mořskými pannami mezi obyvateli
Mostecka mimořádný zájem, dokazuje sehnutý muž v podvlíkačkách, stejně
jako muž stojící před ním, chystající se rovněž svléknout kalhoty.
Důkazem že v tomto případě poptávka několikanásobně převyšovala
nabídku, je muž otočený zády a tahající ze zadní kapsy šrajtofli. Kdo
byl nakonec tím šťastným, se již s největší pravděpodobností nikdy
nedozvíme. Z jistotou lze vyloučit kromě malé dívenky, snad ještě
chlapce okusujícího si nehty.
