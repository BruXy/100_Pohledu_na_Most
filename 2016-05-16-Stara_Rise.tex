% Zdroj: https://www.idnes.cz/zpravy/sto-pohledu/100-pohledu-na-cesko.A160512_111056_domaci_hro/diskuse/3
% Datum: 16. 5. 2016 9:53
% Foto: https://1gr.cz/fotky/idnes/16/052/org/JB6337be_v3f02.jpg
% Body: +108 -2

\chapter{Most~-- Čepirohy}

V~tamním klubu jsem prožil část své fotbalové kariéry. Zpočátku jsem moc šancí
prokázat svůj talent, od trenéra nedostával. Vlastně jsem do žádného zápasu
nikdy nenastoupil.

Moje chvíle přišla, až když většinu týmu zdecimovala epidemie příušnic. Trenér
mi kladl na srdce, abych se za všech okolností držel co nejdál od naší branky.
Což jsem svědomitě plnil. Zatímco většina hráčů se neustále vyskytovala kolem
míče, já se od něj snažil držet co nejdál. Jenže pak to přišlo.

Z~klubka hráčů vylétl míč a zamířil si to rovnou ke mně. Docela smůla, že to
dopadlo jinak, než bych si přál. Ve snaze o~zpracování míče, jsem kolenem
trefil svůj vlastní obličej, zatímco míč mnou nezasažen pokračoval do zámezí.
Jak k~tomu došlo, úplně přesně nevím, protože úder byl tak silný, že jsem na
okamžik pozbyl vědomí.

Když jsem se vrávoravě zvedl, na hřišti i kolem něj panovalo bujaré veselí.
A~když pak ke mně přiběhl rozhodčí a ukázal mi červenou kartu za nebezpečnou hru
vysokou nohou, nadšení nebralo konce. To bylo asi poprvé, kdy diváci odcházeli
spokojeni, i když jsme prohráli. Od toho dne, jsem za rozhodnutého stavu
nastupoval téměř pravidelně na posledních několik minut. Trenér říkal, že lidi
se mají fotbalem hlavně bavit a já, že jsem na to přesně ten typ.


