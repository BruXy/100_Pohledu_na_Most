% Zdroj: https://www.idnes.cz/zpravy/sto-pohledu/100-pohledu-na-cesko.A160704_104224_domaci_jav/diskuse/2
% Datum: 4. 7. 2016 11:50
% Foto: https://1gr.cz/fotky/idnes/16/063/org/JB643fd5_V5E31.jpg
% Body: +80 -1

\chapter{Vojáci 35. pěšího pluku z Plzně}

Na vzácném snímku je zachycen vůbec první kostel, který padl za oběť těžbě uhlí
na Mostecku. Že nakonec dojde na celé město, tehdy ještě nikdo netušil. Na
fotografii je zaznamenána i událost o~které už dnes málokdo ví. Jsou na ní
zachyceni vojáci 35. pěšího pluku z~Plzně, kteří byli povolání do Mostu, aby
odstřelem zredukovali stavy přemnožených muflonů v~pánevní oblasti.

Že se tento záměr tak ne zcela povedl, svědčí rakev s~jedním z~nich. Po této
nešťastné události byli vojáci staženi do svých domovských kasáren a vedení
dolů se snažilo problém se sudokopytníky řešit po svém.

Tehdy přišel někdo s~nápadem, že by bylo možno snížit stavy zvěře nasazením
páru bengálských tygrů. Ty zcela nezištně zapůjčil za městem zimující cirkus
Klucký\footnote{Existuje i slavnější cirkus, jehož název se píše jako {\em
Kludský}. Cirkus {\em Klucký} o kterém píši já, jen umně využil podobného názvu
ke své snadnější propagaci.}. Bohužel ani tento nápad se zjevně neosvědčil,
protože stavy muflonů se prakticky vůbec nesnižovaly. Ba spíše naopak. Zato
začali citelně ubývat zaměstnanci. Pravděpodobně proto, že neuměli tak rychle
běhat. Údajně v~tom svou roli sehrávalo i měkčí maso. Na tento hloupý
experiment nakonec doplatil i samotný cirkus, protože navrátivší se tygři,
přestali být vhodní pro vystupování v~manéži. Neustále totiž po svém krotiteli
i ostatních zaměstnancích zcela nezakrytě mlsně koukali.

Problém s~muflony se nakonec podařilo vyřešit vcelku jednoduše, a to přísným
zákazem odhazování svačin všemi zaměstnanci dolů. Muflonům tak pro nedostatek
plnohodnotné stravy nezbylo nic jiného, než se vrátit na svahy Krušných hor.

