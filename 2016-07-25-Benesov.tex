% Zdroj: https://www.idnes.cz/zpravy/sto-pohledu/100-pohledu-na-cesko.A160725_094437_domaci_hro/diskuse/3
% Datum: 25. 7. 2016 10:45
% Foto:  https://1gr.cz/fotky/idnes/16/063/org/JB6413f9_vv4h02.jpg

Tak v tomto případě se asi s panem redaktorem neshodneme. Nejedná se o
žádnou zříceninu, ale o tzv. "letní" kostel ve starém Mostě. Zda byl
kostel v této podobě koncipován již od prvopočátku, se mi doposud
nepodařilo dohledat. Pokud vím, další takto revolučně pojatá stavba se
na našem území nenachází. Pomineme-li nedostavěný gotický chrám Panny
Marie v Panenském Týnci, který však nikdy nesloužil svému účelu.
Fotografovi se kromě kostela podařilo zachytit také farářovu hospodyni
Anežku, topiče Františka s nůší dřeva na zádech a jednu ze sester řádu
magdalenitek přezdívaných též "bílé paní". U té není vzhledem k tomu,
že je otočena zády, totožnost s určitostí prokázána. Většina historiků
se však přiklání k názoru profesora Chtivého, že se podle tvaru pozadí
jedná o sestru Eulálii. Ačkoli se kostelu říkávalo "letní", byl až na
vzácné vyjímky využíván celoročně. A to i v chladné podkrušnohorské
zimě. To se pak musel topič František pořádně otáčet. Jen zcela
ojediněle se stávalo, že byl obřad kvůli nepříznivému počasí posunut,
či přeložen. A to pouze v případě sněhové kalamity, když se panu
faráři nepodařilo včas proházet se sněhem k oltáři. Na snímku kostela
už nevidíme původní zdobená okna. Ta byla demontována, protože při
jejich mytí trpěla hospodyně Anežka závratěmi. S údržbou oken na zemi,
pak už neměla žádný problém. Jejich absence stejně měla na klima v
samotném kostele jen zanedbatelný vliv. Sestry z řádu magdalenitek
pečovaly o pana faráře převážně v nočních hodinách, nebot' hospodyně
Anežka byla šťastně vdaná a ten její sloužil u c.k. státních drah
jenom ranní. Škoda, že i tento ojedinělý architektonický klenot musel
ustoupit povrchové těžbě hnědého uhlí. Nicméně posloužil alespoň jako
inspirace pro výstavbu různých amfiteátrů a letních kin na celém území
naší vlasti.

