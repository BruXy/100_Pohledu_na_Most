% Zdroj: https://www.idnes.cz/zpravy/sto-pohledu/100-pohledu-na-cesko-sedlec-u-kutne-hory.A160915_090247_domaci_jj/diskuse/2
% Datum: 15. 9. 2016 10:01
% Foto: https://1gr.cz/fotky/idnes/16/081/org/JB6523fb_aH.jpg
% Body: +118 -0

% Pro případné kritiky podotýkám, že následující řádky mají nulovou
% literární hodnotu a že jsem si této skutečnosti vědom. Těm, kterým se
% mé předchozí příspěvky nelíbily, nedoporučuji číst dál, protože lepší
% už to opravdu nebude.

\chapter{Kulturní dům Repre}

Na snímku lze vidět budovu Repre v~Mostě. Obrovský kulturák se spoustou sálů,
sálků a kluboven. A~taky s~restaurací a kinem Oko. O~kterém už jsem se v~jiné
souvislosti zmínil.

Jednou nám soudružka učitelka přinesla do třídy nabídku různých zájmových
kroužků co se tam pořádaly a do kterých i my prý můžeme chodit. Já jsem zkusil
nejprve nějaký o~vesmíru. Byl jsem tam asi dvakrát. Pořád jsme si psali do
sešitu jenom nějaká čísla a vzorce. Jako ve škole. Dalekohled, kterým by se
daly pozorovat hvězdy a planety, nebo při zatažené obloze alespoň obnažené ženy
v~okolních domech, nám neukázali ani na fotce.

Tlaku matky která chtěla abych navštěvoval kroužek šití, jsem naštěstí odolal.
I~když argumentovala tím, že až ze mě bude doktor a budu operovat, může se mi
to hodit. Já se ale chtěl v~té době stát popelářem. Líbilo se mi, jak umí
koulet dvě popelnice najednou a že se můžou celý den zadarmo vozit na
stupačkách. Pak jsem taky chvíli navštěvoval nějaký kroužek přátel sovětské
literatury, protože se tam přihlásila Helenka. Když jsem zjistil, že ty knížky
nejsou o~indiánech a Helenka na mě kašle, vykašlal jsem se zase já na ten
kroužek.

Nakonec mi Jarda, co bydlel pod náma a už chodil do učení, poradil z~těch
kroužků ještě jeden. Že je to o~holkách. Jako, že co mají pod oblečením, a
tak\dots{} To mě poslední dobou začalo docela vážně zajímat a tak jsem se tam
moc těšil. Zbytečně. Ukázalo se, že se to jmenuje podobně, ale je to o~psech.
Kroužek mladých kynologů\footnote{\dots{}a nikoliv gynekologů}. Nebo tak nějak.
Všichni co tam přišli, měli sebou nějakého psa. Já měl doma akorát křečka a při
představě, že bych ho hledal na cvičáku v~trávě, jsem odešel zklamaně domů.
Nebyl jsem smutný, že nemám psa, ale že to nebyl ten kroužek o~kterém mi
vyprávěl Jarda.

% Mě by zajímalo, jestli to není nějaká známá Mostecká osobnost, že je na
% několiká fotkách z Mostu....

% Máte pravdu. Ale jméno tohoto mosteckého občana se nám bohužel těžko podaří
% zjistit. Co je ale jisté je to, že snímek který je mylně označován jako
% Kolín, je novějšího data než ten dnešní a je důkazem, jak se lze dobře
% zvoleným podnikatelským záměrem vypracovat. Pán s trakařem na dnešním snímku,
% zřejmě pracoval jako uklízeč výkalů po psích spřeženích. Těmi následně hnojil
% pronajatý kousek políčka a výsledek vidíme na "kolínském" snímku. Místo
% čepice buřinka, místo trakaře žebřiňák a před ním hromada výpěstků. Ještě k
% psovi na fotce. Když se podíváme na jeho vyplazený jazyk, jednoduše nám
% dojde, kde se vzalo rčení "má chování řeznickýho psa".
