% Zdroj: https://www.idnes.cz/zpravy/sto-pohledu/sto-pohledu-na-cesko.A160804_160619_domaci_hro/diskuse/4
% Datum: 5. 8. 2016 10:19
% Foto:  https://1gr.cz/fotky/idnes/16/073/org/JB64c7a4_V5E24.jpg

Smetanovo náměstí v~Mostě. Budova napravo od věže je Městské divadlo.
Po roce 1948 se přejmenovalo na Divadlo pracujících. Jeho slavnostní
otevření proběhlo 30.9.1911. Architektem byl Alexander Graf. K~jeho
odstřelu došlo 20.10.1982. Ale než ho stačili zbořit, několikrát jsem
ho navštívil se školou v~rámci kulturního vzdělávání. Užili jsme si
tam ovšem i spoustu legrace. Jednou soudružce učitelce z~"béčka"
strašně vadilo, že jsme před představením při hře na honěnou skákali
po křeslech. Ludvíkovi, který dojížděl z~nedaleké obce řekla, že není
někde v~Kotěhůlkách v~kině, čímž ho docela nakrkla. A~tak když sál
potemněl, odměnil jí občas střelou z~flusačky z~krajónu, do které
používal rýži jako střelivo. Soudružka pak po každém úspěšném zásahu
slíbila někomu dalšímu ve svém okolí třídní důtku. Trefit nějak
zásadněji některého z~účinkujících, se Ludvíkovi pro velkou vzdálenost
moc nedařilo. Samotné dění na jevišti, mě tehdy nijak zvlášť
nezaujalo. Pořád tam běhali v~ušankách, říkali si soudruhu a dělali že
staví koleje pro bajkalsko-amurskou magistrálu. Když to pak s~námi
naše soudružka učitelka po představení probírala, tak řekla, že když
se nebudeme učit, dopadneme stejně. Jednoho toho soudruha, co se mu
nechtělo moc pracovat kvůli tomu vyloučili ze strany, načež ho za to
opustila manželka a vzala si jeho uvědomělejšího kolegu. O~hraní
divadla jsem také nějaký čas přemýšlel. Viděl jsem to v~té době jako
jedinou možnost, jak dát holce pusu a nedostat po hubě. A~tak když se
ve škole dělal nábor do dramatického kroužku, zvedl jsem ruku. Ta
soudružka co to vedla, ale mé nadšení pro herectví rychle zchladila.
Prý mi není vůbec rozumět jak huhňám a že když by osvětlovač zamířil
reflektor na můj zlatý zub, hrozil by divákům v~prvních řadách zánět
spojivek. Ale že i pro mě se občas taky něco najde. Pochopil jsem, že
z~líbání na jevišti nebude nic a představa, že bych třeba hrál
v~prvním dějství pařez a v~tom druhém mě vykopali, mi vzala veškerou
chuť v~herectví pokračovat.
