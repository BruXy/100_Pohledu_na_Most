% Zdroj: https://www.idnes.cz/zpravy/sto-pohledu/100-pohledu-na-cesko.A160512_123406_domaci_hro/diskuse/2
% Datum: 26. 5. 2016 10:13
% Foto: https://1gr.cz/fotky/idnes/16/052/org/JB634e5a_01.jpg
% Body: +174 -1

\chapter{Unikátní most přes Bílinu}

Na snímku vidíme stavbu unikátního mostu přes řeku Bílinu v~Mostě, jehož část
vidíme v~pozadí. Snímek pochází z~doby, kdy byla ještě řeka ve svém korytu a
nebyla znečištěna. Ženy v~ní praly prádlo a myly své více, či méně špinavé
potomky.

Moje generace zažila již jen černou páchnoucí stoku, do které byl vypouštěn
veškerý odpad především z~Chemických závodů v~Záluží. Zlidovělá píseň {\em
Okolo Mostu voděnka teče, kdo si k~ní přičuchne, chytaj ho křeče}, vedla po
dlouhé týdny okresní hitparádu. Ale na singlu stačilo vyjít jen několik set
prvních výlisků, načež byla na podnět shora, další výroba zastavena. Některé
pasáže textu, údajně vrhaly špatné světlo na práci stranického vedení, a to
nejen na okrese. Z~dlouhohrající desky, byla skladba vypuštěna úplně a
nahrazena méně závadnou {\em Líbíš se mi bez botek, podprsenky, kalhotek},
s~textem mapujícím prostředí tehdejších svazáckých brigád.

Pamatuji několik případů, kdy koryto řeky nebylo schopno pojmout zvýšené
množství protékající tekutiny a tato se rozlila v~níže položených částech
města. O~tom, zda byly tyto lokální záplavy způsobeny zvýšeným množstvím
vypouštěného odpadu, či vlivem atmosferických srážek, se vedly pokaždé vášnivé
debaty. V~době, ve které je řeka zachycena na fotografii, se dařilo mimořádně i
lodní dopravě. Jenomže v~důsledku povrchové těžby uhlí, došlo nejen k~odklonění
původního toku řeky, ale některé její úseky bylo nutno vést potrubím, což
citelně omezilo její splavnost. Proplouvat tak mohla už jen užší plavidla
s~nízkým ponorem. Tedy kanoe a kajaky. Majitelé pramic tak měli smůlu. Když se
ale v~potrubí, jednomu z~kajakářů vzpříčilo pádlo při pokusu o~eskymácký obrat
a on i několik dalších muselo čekat uvnitř dlouhé hodiny, než byla řeka
vypnuta, byl nakonec i těmto plavidlům průjezd potrubím zakázán.

