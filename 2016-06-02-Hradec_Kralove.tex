% Zdroj: https://www.idnes.cz/zpravy/sto-pohledu/100-pohledu-na-cesko.A160512_130155_domaci_hro/diskuse/2
% Datum: 2. 6. 2016 11:26
% Foto: https://1gr.cz/fotky/idnes/16/052/org/JB634e75_002.jpg
% Body: +163 -1

\chapter{Štramák v~klobouku}

Ten štramák na snímku, je nevlastní otec pana Otrlého, který měl trafiku
v~Mostě, v~Okružní ulici. My kluci jsme ho měli rádi, protože mluvil sprostě.
Stačilo hodit kámen na jeho dřevěnou budku a dokázal vtěsnat do jedné jediné
věty i několik výrazů pro pohlavní orgány. Z~celé škály jeho výrazů, patřily ty
z~oblasti zoologie, mezi nejslušnější. Kulhal na obě nohy a k~tomuto zranění
přišel v~průběhu druhé světové války. Jeho verzi, že mu nohy omrzly po té, co
se po celý leden ukrýval v~žitném poli před fašisty, ale nikdo nevěřil. Ve
skutečnosti ho do nich střelil hajný Ouhoř, aby měl jistotu, že už mu nebude
běhat za ženou. Ta mu ale nakonec stejně utekla, protože by musela být
střelená, aby s~ním zůstala. Což nebyla.

Pan Otrlý svou jadrnou mluvu hojně využíval i při prodeji tabákových výrobků a
tiskovin. Hlášky typu: \uv{Tady máš ten Dikobraz, do prdele si ho vraz},
\uv{Jedny tvrdý Sparty, dám ti prdel na rty}, \uv{Rychle si ty Startky ber,
hlavně se z~nich neposer}, a mnoho dalších, byly pro něho typické a byl jimi
vyhlášený široko daleko. Neopomněl přidat něco k~dobru také ženám: \uv{Žena a
móda, královno, je ti stejně na hovno}, \uv{Květy a Vlastu, už blbneš
z~chlastu}. A~také dětem: \uv{Za korunu lízátko, posereš se zakrátko}.

Pan Otrlý byl natolik populární, že o~něm psali i v~okresních novinách. Byla
tam fotografie a v~článku pan redaktor vyzdvihoval jeho lidovost a švejkovský
humor. Ovšem jednoho dne se to celé nějak zvrtlo. To když u~jeho stánku nechal
zastavit svou Tatru 603 krajský tajemník strany, s~úmyslem zakoupit si Rudé
právo. Trafikant podávaje mu noviny, aby nezůstal nic dlužen své pověsti
pronesl: \uv{Jedno rudý, račte si ho strčit do prdele vašnosto}. A~aby toho
nebylo málo, přidal otřepaný vtípek: \uv{A~víte pane, proč ho dělaj tak velký?
No, aby nebylo vidět na toho blbce co to čte.} Hned druhý den zůstal stánek
zavřený a do týdne byl pryč. Zmizel i pan Otrlý. Proslýchalo se, že prodává
v~kantýně. Někde na Borech.

