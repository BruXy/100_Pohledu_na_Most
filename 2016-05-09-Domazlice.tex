% Zdroj: https://www.idnes.cz/zpravy/sto-pohledu/100-pohledu-na-cesko.A160509_085613_domaci_jj/diskuse/4
% Datum: 9. 5. 2016 10:17
% Foto:

Most. Ženy na snímku jdou právě z obchodu, kterému místní neřekli
jinak, než stáčírna. A to i později, když tam visela cedule "Lahvové
nápoje". To jsou ta vrata zcela vlevo. Jednou přišel otec domů
nečekaně už za světla a sebral mi šněrovací míč, co když s ním dáte
hlavičku, máte to vázání obtisknuté na čele ještě druhý den.

Potom mi nakázal, abych vzal kárku a do ní naložil všechny prázdné
láhve od piva a jel je odevzdat do stáčírny. Utržené peníze mu pak mám
odnést do hospody, kde na ně bude čekat. Kárka byla plná až po okraj.
Naštěstí byla velká a měla sajtny. Když jsem ale přijel před krámek,
měli zrovna přejímku zboží. Naštěstí se jedna hodná paní, co tam
koukala z okna v přízemí nabídla, že u ní můžu kárku nechat a že lahve
za mě odevzdá a já si jen přijdu pro peníze. Nabídku jsem s potěšením
přijal a radostně odběhl vedle do parku, sledovat pohyb nožek a
rašících ňader dívek skákajících panáka. Když jsem se asi za hodinu u
té paní zastavil pro peníze, dávala mi tři koruny a říkala, že všechny
láhve byly popraskané a odmítli jí je vzít. A že ty tři koruny mi dává
ze svého. Bylo mi té obětavé paní líto a ani ty tři koruny jsem si od
ní nevzal. Vždyť ta nebohá žena si ani nepamatovala, že jsem u ní
nechal i kárku s pumpičkou. Tak na tom byla špatně. Když jsem to pak
doma celé vyprávěl matce, řekla, že jsem blbej jak daleko vidím.

Druhý den ráno jsem jí pak slyšel jak říká sestře, že otec nechal v
hospodě sekeru. Možná to tak bylo lepší. Ještě by s ní mohl sobě, nebo
někomu dalšímu v opilosti ublížit.
