% Zdroj: https://www.idnes.cz/zpravy/sto-pohledu/100-pohledu-na-cesko.A160512_123548_domaci_hro/diskuse/3
% Datum: 30. 5. 2016 9:34
% Foto: https://1gr.cz/fotky/idnes/16/052/org/JB634e6b_001.jpg
% Body: +139 -3

Třetí náměstí v Mostě. Je vidět, že už v předminulém století sloužilo
jako důležitý dopravní uzel. Jen povozy, které vidíme na snímku, byly
později nahrazeny o dost rychlejšími autobusy. V podloubí úplně
vpravo, bývala hospoda Řemeslo. Tu s oblibou navštěvoval i můj otec.
Pamatuji, jak tam jednou po mnoha letech malovali. Už bylo třeba. Zdi
byly prohuleny až na cihly a když chtěli složit záclony, tak se jim
zlomily. Ty dny, jsme viděli otce i za světla. Nesl to těžce ale matka
byla ráda, že konečně uvidí, jak se sestrou vypadáme. Aby se cítil
jako doma, poslala mě s brašnou pro lahváče a sama mu uvařila jeho
oblíbenou rumovou polévku a pivní guláš. Měla očividně starost, aby v
této pro něho neobvyklé situaci, neutrpěl nějaké trauma. A tak já jsem
měl říkat, že se dobře učím, ve škole nezlobím a pionýrský šátek mám v
čistírně. Sestra zase, že má velké břicho protože jí chutná. Pro
jistotu se ale neměla ukazovat moc ze strany a moje žákovská skončila
ve sklepě pod uhlím. Matčiny obavy se však ukázaly jako liché. Otec po
několika pivech a třetím talíři polévky, kdykoli zahlédl sestru,
objednával panáky pro celou hospodu v domnění, že se jedná o servírku
a mě měl za liftboye a chtěl odvézt do třetího. Když pak spokojeně
usnul na stole, nikomu to nějak zvlášť nevadilo. Vlastně ani to, když
v Řemesle skončili s malováním a vše se vrátilo do zaběhnutých kolejí.

