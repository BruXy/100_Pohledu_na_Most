% Zdroj: https://www.idnes.cz/zpravy/sto-pohledu/100-pohledu.A170112_093922_domaci_jj/diskuse/2
% Datum: 12. 1. 2017 10:39
% Foto: https://1gr.cz/fotky/idnes/17/011/org/JB68679f_v5e38.jpg
% Body: +94 -0

Ani noha na prosluněném náměstí...Ve starém Mostě pochopitelně. Právě
tato skutečnost mě přiměla neplánovaně přispět se svými poznatky do
této diskuse. Vysvětlení proč tomu tak 11.\,9.\,1911 bylo, je vcelku
jednoduché. Bylo to proto, že se všichni obyvatelé města zúčastnili
pohřbu legendárního mosteckého rodáka básníka Fráni Trámka. Ten den
byl ovšem také významný tím, že nikdo nikomu nic neukrad. A když, tak
to byl určitě někdo z přespolních. Zemřel poeta, milovník života.
Milovník žen, milovník mužů, milovník domácích i hospodářských zvířat.
Prožil dlouhý a plodný život během něhož splodil několik nemanželských
dětí, přibližně tolik básnických sbírek a zástup věřitelů. Jeho hlášky
"žízeň je věčná" a "dluhy se neplatí" posléze zlidověly. Že bude
básníkem se rozhodl už coby nemluvně, když ležel v zavinovačce a
maminka původem odněkud z východního Slovenska mu broukala ukolébavku:
"Išiel zajac cez potok, stratil vajcia aj kokot". Na závěr připojím
ještě malou ukázku z jeho díla. Tentokrát ze sbírky "Měl jsem jí moc
rád, než mě začla srát.

Tvářil se tak odhodlaně, když do kalhotek strkal dlaně.

Jenže když mě rozdělal, usnul, čímž to podělal.

Až příště ovládnou tě chtíče, dej ruce pryč od mojí ....!*

Však hlavu vzhůru nezoufej, vraz ho do úlu a zabouchej.

*Zde se patrně vyřádil tiskařský šotek a některá z písmenek se kamsi zatoulala. Snad měl mistr na mysli slovo "spíže", ale s jistotou to tvrdit nemohu.

