% Zdroj: https://www.idnes.cz/zpravy/sto-pohledu/100-pohledu-na-cesko.A151201_123809_domaci_jav/diskuse/2
% Datum: 25. 4. 2016 8:27
% Foto:  https://1gr.cz/fotky/idnes/16/023/org/JB619e1b_ne.jpg

Fotografie nám ukazuje, jaká pohoda panovala na kolonádě ve starém
Mostě. A to jak mezi zaměstnanci, tak i lázeňskými hosty. Málokdo dnes
ví, že za zánikem tohoto mimořádně významného města, vlastně může pod
ním plánovaná stavba metra. To mělo zpříjemnit místním horníkům,
energetikům a chemikům cestu za prací. Bohužel už při počátečních
ražbách vyšlo najevo, na jak kvalitním uhlí město leží a jeho dny tak
byly bohužel sečteny. Rozhodnutí tehdejší strany a vlády umocnil i
nedostatek uhlí při stávajícím chladném počasí a skluz v plnění plánu
pětiletky.

Plánovaná stavba metra se pak z tohoto důvodu přesunula do Prahy, u
které byl zkušebními vrty výskyt uhlí, či jiné strategicky významné
suroviny vyloučen a k jejímuž bourání tak nebyl důvod. Do Prahy se
také přesunuly některé mostecké památky a významní straničtí
představitelé. Zbytek obyvatel byl přestěhován zhruba o kilometr dál
do nově vystavěných panelových domů. Ti pokud nezemřeli, tam žijí
dodnes.
