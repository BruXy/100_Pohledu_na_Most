% Zdroj: https://www.idnes.cz/zpravy/sto-pohledu/100-pohledu-na-cesko.A160609_085911_domaci_jj/diskuse/3
% Datum: 9. 6. 2016 10:44
% Foto:  https://1gr.cz/fotky/idnes/16/052/org/JB634e85_021.jpg
% Body: +144 -2

To budou Kopisty. Obec, sotva tři kilometry od starého Mostu. Kvůli
těžbě uhlí, jí v sedmdesátých letech minulého století potkal stejný
osud, jako toto královské město a zanikla. Tam bydlela matčina sestra,
teta Emča. Skoro jsem tam nejezdil, protože jen pořád něco vařila,
pekla, hned to jedla a tak pořád dokola. A bylo to znát. Nikdy se
nevdala. Snad proto, aby se nemusela o jídlo dělit. Když stála ve
frontě na maso, vypadala jako sloní kel v obilí a při chůzi se jí obří
prsy vlnily, že připomínala velblouda při probíhání slalomových
branek. Mého otce neměla ráda. Všude o něm rozhlašovala, že si jako
kojenec spletl matčin prs s pípou a už to nechtěl měnit. Jednou, když
jsem přišel z venku, byla zrovna teta Emča u nás a opět na něm
nenechávala nit suchou: "A jak chlastá. Ten chlap má snad dutý nohy,
co se do něj vejde piva. Snad ten chudák nebude po něm," prohlásila a
starostlivě si mě změřila pohledem. "Jó holka, jestli bude mít dutý
nohy po fotrovi, jak říkáš, to poznáme, až vyroste. Ale že má dutou
hlavu, to pozoruju už ted',"pronesla cituplně matka a pohladila mě po
vlasech. "Měl bys víc jíst, jinak tě nevemou na vojnu," pokračovala
teta. "Přinesla jsem ti bábovku ale když jsi dlouho nešel, tak jsem jí
snědla. Na jídlo musíš chodit včas," dodala a prohledávala nám
ledničku. Po její prohlídce vypadala zklamaně. Náladu jí nezlepšila
ani následná kontrola spíže. Nejspíš zalitovala, že k nám vůbec
chodila a chystala se k odchodu. Poslední ránu jí zasadil přiopilý
otec, se kterým na sebe narazili v předsíni, když místo pozdravu
prohlásil, že s tou almarou, co vedle ní stojí, vypadají jako
dvojčata. Tetu Emču jsem u nás pak docela dlouho neviděl.
