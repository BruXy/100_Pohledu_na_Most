% Zdroj: https://www.idnes.cz/zpravy/sto-pohledu/100-pohledu-na-cesko.A161027_091306_domaci_hro/diskuse/2
% Datum: 27. 10. 2016 10:25
% Foto: https://1gr.cz/fotky/idnes/16/103/org/JB66e3e2_k.jpg
% Body: +82 -0

% Poznámka: Vlastův příspěvek cenzor smazal, ale naštěstí byl
% několikrát zkopírován a vrácen do diskuze i ostatními čtenáři.

Ty moravské vrchy mi tam moc nesedí. Nejspíš to bude chyták. Já si
tipnu Meziboří (německy Schönbach). Městečko v Krušných horách, kousek
nad Litvínovem. Okres Most, samozřejmě. Tam mělo naše učiliště školu.
To je ta dominantní budova vpravo. Taky tam byl internát, ale protože
tam nebyly žádný holky, jezdil jsem domů. Bylo to ve druháku, když nám
třídní 30. dubna oznámila, že máme druhý den ráno naklusat před školu
a pak že půjdeme lesem dolů do Litvínova na náměstí na oslavy 1.máje.
Protože jak dodala, ti soudruzi na té tribuně potřebují cítit tu
podporu zdola. Jako že za nima ta dělnická třída stojí. Mě ovšem
naštvala úplně nejvíc, protože mi řekla, že ponesu transparent, kde
bylo něco jako: "Úkoly zadané nám 14. sjezdem strany, minimálně
překročíme". Když jsem namítnul, že by si tu čest neměl nechat ujít
některý ze svazáků, tak mi řekla že at' koukám držet klapačku a že tak
aspoň nezdrhnu jako vloni. Byl jsem rozhodnutej nejít i za cenu
prů.eru. Jenže... Když mě příští hodinu poslal učitel na elektro do
sborovny pro nějakej motor v řezu, tak když jsem tam vlez nikdo tam
nebyl a vedle z kuchyňky slyším naší třídní: "Jarmilo, už mě to fakt
sere. Doma mi stojí práce a my zas budem blbnout půl dne někde na
náměstí. At' už s tím dou do prdele. Debilové." Když jsem zakašlal,
málem upadla. "Jdu tady pro ten motor. A nic jsem neslyšel." Ona na
to: "Ten transparent ponese někdo ze svazáků, jestli ti to nevadí."
Nevadilo, fakt ne. Nesl ho Kája, co si ho učitel na "Stroje a
zařízení" oblíbil natolik, že mu pokaždý říkal, že je to hlava
otevřená krumpáčem. Kája vypadal spokojeně a snad tomu nápisu i věřil.
A já se tak mohl na náměstí v klidu vypařit jako posledně.

A protože je máj i měsícem lásky, hodí se malá ukázka od mostecké
básnické legendy Fráni Trámka z jeho sbírky "Měsíc pod dekou".

Díval se mi do očí
a říkal slova vzletná,
když zašeptla jsem vem si mě,
tak řek mi, že sem běhna.

Onehdá jsem ve výtahu její vůni nasák,
když jsem jí plác po prdeli,
řekla že jsem prasák.
