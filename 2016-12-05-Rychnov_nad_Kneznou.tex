% Zdroj: https://www.idnes.cz/zpravy/sto-pohledu/100-pohledu-na-cesko.A161205_095338_domaci_jj/diskuse/2
% Datum: 5. 12. 2016 11:09
% Foto:  https://1gr.cz/fotky/idnes/16/113/org/JB679c6e_OK.jpg
% Body: +176 -1

Tímto snímkem se po čase vracíme do starého Mostu. Konkrétně na jedno
z mnoha tamních náměstí. Za pozornost stojí především švihák stojící
vpravo od pouličního osvětlení. Měl by to být totiž jistý Arnošt
Chlípný, vyhlášený svůdce žen. Ne náhodou se o něm tradovalo, že by si
to rozdal i s dámským kolem, kdyby mu neujelo. V dobových policejních
spisech byl objeven i jeden dost kuriózní případ. Jeho příběh se začal
odvíjet v září 1883 (i když ve skutečnosti začal už o nějakých sedm let
dříve), kdy se do sebe pustili řezník Bůček a pekař Kvásek. Řezník
Bůček vyhrožoval Kváskovi že ho namele do jitrnic a pekař Kvásek zase
Bůčkovi sliboval, že mu rozválí koule do listovýho těsta. Ten konflikt
mezi nima vzniknul proto, že při zápisu svých synů do první třídy
obecné školy oba dva zjistili, že mají prakticky totožný kluky a tak
si dali dohromady, že mu ten druhej musel teda zákonitě chodit za
ženou. V Mostě to tenkrát bylo téma číslo jedna a pro wachtmeistra
Pytla, který dostal případ na starosti vskutku tvrdý oříšek. O nějakém
DNA si tenkrát mohl nechat jenom zdát. On měl k dispozici pouze metodu
DN (Doznání Nevěrnic). Variant DN bylo hned několik. U zbožnějších žen
se využívalo metody falešného kněze, kdy žena v domnění že je u
zpovědi vyklopila všechno policejnímu figurantovi. Další méně sporný
způsob byl, když se zpovídané ženě nabídlo něco moc hezkého na sebe.
To dost často nejedné rozvázalo jazyk. Metoda, kdy byl vyšetřované
ženě nabídnut sex přímo na četnické stanici a její souhlas byl brán
jako přiznání se k nevěře, byla u četníků velmi oblíbena obzvláště
pokud se jednalo o ženu atraktivní. To se mnohdy vyšetřování
zúčastnila celá četnická stanice. S kterou z variant se tehdy podařilo
vyšetřit případ wachtmeistrovi Pytlovi se už asi nedozvíme, ale jisté
je že jak paní Bůčková, tak paní Kvásková označily za otce chlapců
Arnošta Chlípného. Ten, kromě toho že dostal od obou živnostníků
brutální nakládačku a neřeklo se mu jinak než Sperminátor, už si v
Mostě ani nevrznul.

