% Zdroj: https://www.idnes.cz/zpravy/sto-pohledu/100-pohledu-na-cesko-jindrichuv-hradec.A160620_092833_domaci_jj/diskuse
% Datum: 20. 6. 2016 10:33
% Foto: https://1gr.cz/fotky/idnes/16/062/org/JB63f7cc_rad.jpg
% Body: +99 -1

\chapter{Rozvodněná Bílina u Českých Zlatníků}

Na tomto vzácném snímku je vidět, co dokázala rozvodněná řeka Bílina. To za
řekou, jsou České Zlatníky. Obec jen kousek za Mostem. V době povodní, byla
tato zcela odříznuta, a jediné možné spojení, bylo z Mostu po vodě. V pozadí
bohužel nevidíme vrch Zlatník. Pravděpodobně proto, že tehdejší fotoaparáty
neměly takový dosah. Ovšem na vině může být i mlžný opar.

V Českých Zlatníkách, bydlel i můj spolužák Lojza, co sedával v poslední lavici
u okna. Ten si jednoho jarního dne, přinesl k svačině syrečky. Takové, co je
jeho otec nakládal do piva a následně je nechával několik dalších týdnů zrát.
Tyhle byly uleželé opravdu důkladně, a tak když Lojza svačil, my ostatní jsme
nestačili větrat.

Zazvonilo a soudružka učitelka přikázala okna zavřít, aby její výklad nebyl
rušen hlukem z ulice. Na argumentaci mnohých z nás, že je ve třídě pořád hrozný
smrad, opáčila, že syrečky jsou zdravé. To ovšem neměla říkat. O přestávce jsem
zašel za Lojzou s krajícem chleba a požádal jsem ho, aby mi na něj taky trochu
těch syrečků dal, že bych moc rád ochutnal. Ochotně tak učinil, protože i jemu
samotnému začaly lézt pomalu i ušima a odešel na záchod kouřit.

Já mezitím přemístil tuto zrající hmotu z chleba, na spodní desku učitelčina
stolu. Po několika dalších minutách, vyučování pokračovalo. Bylo zřejmé, že
celou tu hodinu byla soudružka učitelka značně nervózní a nesoustředěná. Chvíli
u stolu neposeděla a dokonce sama otevírala okna. Když zazvonilo, očividně si
oddechla. Ale než odešla, zavolala si Lojzu k sobě a řekla mu, že ať si těch
syrečků jí doma s tatínkem kolik chce, ale do školy, ať už je v žádném případě
nenosí. Jinak že by se mu mohlo stát, že mu je zabaví a dostane je až na konci
školního roku.

