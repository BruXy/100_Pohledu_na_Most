% Hyper-linky

\useURL[mymail][mail:bruchy@gmail.com][][bruchy@gmail.com]

\useURL[10pohleduVlasty]
    [https://www.idnes.cz/zpravy/domaci/100-pohledu-na-cesko.A161229_154804_domaci_jav]
    [][10 POHLEDŮ VLASTY BENEŠE: Jakpak bych to nepoznal, Most!]

\useURL[100pohleduNaCesko]
    [https://zpravy.idnes.cz/sto-pohledu-na-cesko.aspx]
    [][100 Pohledů na Česko]


\chapter{Doslov}

Pokud vám jméno {\em Vlasta Beneš} nic neříká, možná si kladete otázku: {\bf Opravdu je
tato publikace o~Mostu?} A~odpověď je: {\bf NENÍ!}

Vše začalo v~roce 2016, kdy web iDNES.cz začal seriál nazvaný {\em
\from[100pohleduNaCesko]}.  Dvakrát týdně vydávali málo známé nebo nikdy
nepublikované fotografie mnohdy neobvyklých míst a nechali čtenáře hádat odkud
ona fotografie pochází. Často se stávalo, že čtenáři fotku přisuzovali právě
starému Mostu, který by zdemolován, aby ustoupil těžbě hnědého uhlí.

A~několik týdnů po začátku seriálu se v~diskuzi vyskytl i Vlasta Beneš. Nejprve
začal celkem nesměle, ale postupem času se rozepsal víc. Neodradily ho ani
nesouhlasné nářky  nebo stížnosti na chyby v~historických souvislostech. Pan
Beneš psal dál a za několik měsíců si získal mnoho přívrženců a podporovatelů,
ale taky několik nepřátel:

{\it Je ostudou seriózního serveru iDnes, že dovolí na svém diskuzním fóru
zveřejňovat lživé a pavědecké nesmysly různých přemoudřelců, zejména jistého V.
Beneše. Toto anonymní individuum, vydávající se za odborníka na historii, zde
opakovaně publikuje výplody své fantazie, které nemají nejmenší oporu ve
faktech.}

Podporovatelů a přívrženců však bylo podstatně víc, a někteří, mě nevyjímaje,
bez čtení přeskočili celý článek a šli rovnou do diskuze, kde se pozitivně
obodovaný příspěvek pana Beneše vyjímal už na prvním místě. Zastánci pak Vlastu
bránili jak jen mohli~-- na výkřiky jako: \uv{Vy jste ale \uv{pamětník}. Co to
melete za pitomosti?}, odpovídali třeba: \uv{Asi má pravdu, podívej se kolik
lidí mu dalo \uv{+}, ti všichni o~tom něco vědí.}

Mnoho Vlastových přívrženců také volalo po knižním vydání jeho textů. Už tehdy
mi to přišlo jako výborný nápad, ale protože nevím o~nikom, kdo by se tohoto
úkolu zhostil, nakonec po 7 letech jsem se ho ujal já sám.

Tím začala celkem mravenčí práce, protože web iDNES.cz neumožňuje vyhledávat
diskuzní příspěvky podle autora a tak jsem musel projít články ručně. Aby si
fotky v~galerii iDNES.cz nemohl zkopírovat kdejaký čičmunda, implementovali
jejich programátoři důmyslnou ochranu, kterou se mi podařilo prolomit asi za 2
vteřiny (když se v~Chrome na fotku klikne pravým tlačítkem a zvolí Inspect,
jako první se ukáže, že fotka je umístěná na pozadí pomocí kaskádních stylů a
odtud už se dá link jednoduše zkopírovat).

Kvůli ručnímu procházení diskuzí je možné, že mi některý z~příspěvků unikl a
proto prosím o~jejich doplnění ({\tt \from[mymail]}). Přestože pan Beneš psal velice pravidelně, ke
konci roku ho na chvilku odradilo držkování některých jeho odpůrců a tak se na
chvilku odmlčel. Poslední příspěvky se pak objevili v~diskuzi začátkem roku
2017. Tady pak stopa končí, občas je zmíněno, že se mistr přesunul na Facebook,
ale jeho fanklub se mi tam najít nepodařilo.

Zajímavé je, že zatímco příspěvky Vlasty Beneše se v~diskuzi dochovaly, většina
stížností jeho odpůrců nikoliv.

Zadostiučiněním pak byl silvestrovský článek {\em \from[10pohleduVlasty]},
kterým i redakce ocenila jeho literární nadání a zápal pro město Most.

Já jsem se příspěvky dobře bavil třikrát. Poprvé, když jsem je po letech
v~diskuzi zase našel a zkopíroval. Podruhé, když jsem je editoval do formátu
knížky. A~nakonec při korekturách a kontrole sazby hotové publikace.

Opravil jsem překlepy, ale záměrně nechal mnohé nespisovné výrazy. Protože pan
Beneš byl nucen dodržovat pravidla slušné mluvy diskuze vulgární a jadrná
slovíčka cenzuroval, ale do knihy jsem je vrátil zpět,  Dále jsem doplnil
nadpisy a v~místech, kde Vlasta polemizuje s~obsahem nápovědy v~původním článku
jsem jej upravil tak, aby to odpovídalo formátu knihy. Kapitoly jsou řazeny
chronologicky tak, jak články na iDNES.cz vycházely.

\section{Kdo je Vlasta Beneš?}

To nevím a ani se mi to nepodařilo zjistit\dots{} Vlasta Beneš je rozhodně literární
talent, předpokládám, že z~toho co píše i mnohé sám prožil. V~Mostě musel
strávit většinu života a možná z~něj i pochází. Z~častého používání {\tt t'}
nebo {\tt d'} usuzuji, že v~jeho svalové paměť byla naprogramována ještě v~době
psacích strojů a tak by mu možná mohlo dnes být odhadem okolo 70 let.

Ze začátku psal Vlasta příspěvky rovnou, ale později se v~diskuzi přiznal, že
si příběhy připravuje předem a na zveřejněnou fotografii se je snaží po vydání
už jenom \uv{napasovat}.

Protože délka diskuzního příspěvku je technicky omezená, musel mnohdy své texty
upravovat a zkracovat, aby se vešel do limitu maximálního počtu znaků.

\stopcolumns % uzavření za sloupců za chapter/after
