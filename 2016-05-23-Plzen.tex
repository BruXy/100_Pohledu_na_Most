% Zdroj: https://www.idnes.cz/zpravy/sto-pohledu/100-pohledu-na-cesko.A160512_123236_domaci_hro/diskuse/2
% Datum: 23. 5. 2016 11:39
% Foto: https://1gr.cz/fotky/idnes/16/052/org/JB634e64_00.jpg
% Body: +114 -0

Je to smutný pohled na již neexistující Most. Hlavně v této části
města, natáčeli Američané válečný film "Most u Remagenu". Město se
stejně bouralo, tak proč si za pomoc při jeho demolici nenechat ještě
zaplatit. Navíc tvrdou měnou. Přivydělat jsme si ale mohli i my místní
obyvatelé, jakožto členové komparsu. Samozřejmě, že i já projevil
zájem o roli ve filmu a padesát korun k tomu. Na vojáka jsem ještě
neměl výšku a tak jsem se stal členem davu, který byl odváděn vojáky v
německých uniformách. Stačilo jen projít kolem kamery a toho pána, co
seděl vedle ní na takové pohyblivé plošince. Opakovali jsme to ale
několikrát a pořád nebyl spokojen. To jeho "stop", vykřikoval stále
důrazněji a byl stále nervóznější. Poprvé nás zastavil když si všiml
mého luku a čelenky s pérem. Podruhé, když jsem podrazil nohu tomu
pánovi co hrál esesáka, protože samopalem strkal do té paní s holí a
řval na ní něco německy. Když upadl a narazil si o ten samopal žebra,
nadával plynule česky. Nakonec tomu šéfovi vadilo i to, že jsem
zamával do kamery jako sportovci při zahájení Olympijských her.
Opakovalo se znovu i potom, co za mnou jdoucí na smrt, přeze mě
přepadl a po něm někteří další. Ale v tom jsem byl nevinně, protože
jsem si potřeboval zavázat tkaničku. Spokojen vypadal až po té, co se
scéna natáčela bez mé přítomnosti, protože mě nějací dva chlapi drželi
mimo dosah kamer. Když jsem po natáčení přišel k tomu pánovi, co
vyplácel honoráře tak řekl, že mi dá těch padesát korun pod podmínkou,
že už se tam neukážu. A když, zpřeláme mi všechny hnáty. Do filmu pak
samozřejmě zařadili tu nezáživnou scénu beze mě a já se tak na plátně
neviděl.

