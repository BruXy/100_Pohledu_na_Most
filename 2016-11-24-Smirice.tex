% Zdroj: https://www.idnes.cz/zpravy/sto-pohledu/100-pohledu-na-cesko.A161124_092103_domaci_jj/diskuse/2
% Datum: 24. 11. 2016 10:48
% Foto:  https://1gr.cz/fotky/idnes/16/113/org/JB671658_cv10c25.jpg
% Body: +131 -0

\chapter{Nádraží Smiřických}

Soukromé nádraží rodiny Smiřických, které stávalo na periferii starého Mostu.
Vpravo nahoře vykukuje část střechy jejich pivovaru. Na levé straně lze vidět
k~němu náležející nákladové nádraží. Postavit si pivovar a vlastní nádraží při
frekventované trati Ústí nad Labem~-- Cheb se zdálo být velmi dobrým tahem.
Ovšem také nevítanou konkurencí Měšťanskému pivovaru v~Mostě, jehož akcionářům
se nějak podařilo u~vedení c. a k. státních drah prosadit, že žádné vlaky na
tomto nádraží nestavěly. K~tomuto účelu využili služeb nejvlivnějších lobbistů
té doby, kterými byli Darek Malík a Jan Oušek a kteří měli kontakty až na ta
nejvyšší místa ve Vídni.

V~budově nádraží byly v~té levé části kanceláře vedení pivovaru. To bílé kulaté
v~okně nad otevřenými dveřmi, co na první pohled vypadá jako vydlabaná dýně
k~Halloweenu, bude hlava starého pána Smiřického, zakladatele nádraží a
pivovaru. Ten sleduje dění ve skladu naproti a především to, zda jsou tamní
zaměstnanci střízliví. Dveře napravo vedly do pivovarské restaurace Smířa.
V~patře nad ní bydlel správce pivovaru Dieter Wochmelka. Kolo značky Laurin~&
Klement opřené mezi dveřmi patřilo právě jemu. Na něm objížděl hospody v~okolí
a prováděl kontroly kvality podávaného piva.

Kolej nejblíže k~nádraží byla majetkem Smiřických. Ta vzhledem k~faktu, že na
nádraží žádné vlaky nestavěly, sloužila k~rozvozu návštěvníků restaurace do
jejich domovů. Ti měli samozřejmě dopravu zdarma. K~tomuto účelu byla využívána
parní drezína s~nákladovým prostorem pro přepravu pivních sudů. Při převážení
osob se dbalo na přísná bezpečnostní a hygienická pravidla. Drezína byla
vybavena tabulkami s~texty jako: {\em Nezvracejte pokud vlak stojí ve stanici}
či {\em Neplivejte na zem, leží tam váš kamarád}. Před každou jízdou provedl
strojvedoucí, průvodčí, topič a výpravčí v~jedné osobě, kontrolu pasažérů
oprávněných cestovat. Každý z~nich na něho musel dýchnout a pokud z~někoho
nebyl cítit alkohol v~dostatečné míře, byl tento nekompromisně vyloučen
z~přepravy.

