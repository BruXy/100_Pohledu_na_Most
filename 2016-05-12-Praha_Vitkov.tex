% Zdroj: https://www.idnes.cz/zpravy/sto-pohledu/100-pohledu-na-cesko.A160511_152505_domaci_jj/diskuse/5
% Datum: 12. 5. 2016 9:55
% Foto: https://1gr.cz/fotky/idnes/16/023/org/JB619e1d_fit.jpg
% Body: +135 -1

\chapter{Mostecká základní škola}

%Já o~té Praze úplně přesvědčený nejsem.

Ta třípatrová budova vpravo, je totiž naše škola, co stávala v~Mostě. A~právě
z~ní jsme odjížděli na školní výlet do pražské zoo. Když soudružka učitelka
vybírala na ten zájezd peníze a donesl jsem je k~tabuli já, prohodila, že by
některé výtečníky nejraději nechala doma, ale že o~tom bohužel nerozhoduje.

Pak nám povídala o~Praze. Že tam bydlí soudruh prezident, co visí vedle tabule.
A~také další soudruzi co u~tabule nevisí, protože by se na tu zeď nevešli. Ale
že by si viset zasloužili. A~že my tedy v~té Praze nesmíme naší škole udělat
ostudu, protože co by si o~nás pomysleli. Pak nás poučila, jak bychom se měli
v~zoo chovat a že kdyby bylo po jejím, že by klidně pavilón opic vynechala,
protože toho má ve škole pokaždé za celý den dost.

Cesta autobusem proběhla bez problémů. Škoda, že Pepík nechytil to jablko, co
jsem mu hodil přes celý autobus, protože to trefilo soudružku družinářku do
hlavy. Trochu zasmušile se tvářil i pan řidič. Možná kvůli tomu napínáčku na
sedadle. Ani v~samotné zoo se nic zvláštního nestalo, kromě toho, že mi jedna
z~opic nechtěla vrátit láhev s~pitím. Když se s~ní Franta o~tu láhev
přetahoval, sebrala mu ještě hodinky. Kačenka brečela, protože jí štípnul
pštros, když mu chtěla dát pusu a Emil proto, že si sednul na čerstvě natřenou
lavičku a ostatní se mu smáli, že vypadá jako zebra. Soudružku učitelku naštval
jeden papoušek, protože před námi mluvil sprostě. Když ale na poslední chvíli
vytáhla Edu, který si k~ledním medvědům lezl pro čepici, nadávala tak, že by si
s~tím sprostým papouškem mohli podat ruku s~křídlem. Boženku postříkal nějaký
tchoř a hrozně smrděla. A~tak jí soudružka učitelka posadila na zpáteční cestě
vedle Ferdy, protože ten smrděl taky. On ale proto, že dlouho nemohl najít
záchod a když ho konečně našel, bylo už pozdě.

Když nás pak soudružka učitelka předávala v~Mostě rodičům, byla na ní znát
taková úleva, že úplně zapomněla na slíbené poznámky.

