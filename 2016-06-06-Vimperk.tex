% Zdroj: https://www.idnes.cz/zpravy/sto-pohledu/100-pohledu-na-cesko.A160512_130804_domaci_hro/diskuse
% Datum: 6. 6. 2016 10:28
% Foto:  https://1gr.cz/fotky/idnes/16/052/org/JB634e80_x1.jpg
% Body: +128 -0

\chapter{Mostecký pivovar}

%Přiznám se, že dnes mě pan redaktor malinko zmátl tím nadpisem.

Při podrobném pohledu jsem si ale téměř jist, že to nahoře, by mohlo připomínat
zámek je ve skutečnosti mostecký pivovar. Jeho prvopočátky se datují už od roku
1470 a přestal existovat v~roce 1972, kdy byl s~pomocí 25 metráků střeliva
k~tomu 60 tisíc rozbušek srovnán se zemí.

\uv{Tak děti, zítra nás čeká slíbená exkurze do městského pivovaru. V~osm hodin
se sejdeme před školou,} oznámila nám jednoho dne soudružka učitelka. Když jsem
se přihlásil a zeptal se jí, zda by s~námi mohl jít také můj otec, který se
o~tuto problematiku už mnoho let zajímá, tak mi odpověděla, že se jedná
o~prohlídku, která je určena výhradně žákům naší školy. Pak ještě dodala, že
cílem této exkurze je seznámení se s~výrobou piva s~výhledem na jeho další
rozvoj a nikoli na jeho likvidaci.

Když jsem doma oznámil otci, že s~námi nemůže, byl na soudružku učitelku tak
naštvaný, že jí napsal do mé žákovské knížky poznámku a řekl, že jí to mám dát
stokrát podepsat. Nicméně mne vybavil několika prázdnými nádobami a já vyrazil
na exkurzi.

V~pivovaru se nám líbilo, i když to tam trošku smrdělo. Ochutnávat ale směla
jen soudružka učitelka, protože té jediné už bylo osmnáct. Po několika
ochutnávkách si šlapala na jazyk a přestalo jí vadit, že si pivo plním do
nádob. Stejně mi to povolil ten pán, co nás provázel, když jsem mu řekl, pro
koho to je.

Když exkurze končila, soudružka učitelka vrávorala natolik, že ji museli
někteří z~nás podpírat. Pan hlavní sládek, který o~otci už také slyšel, protože
dával jemu i dalším zaměstnancům práci, mi řekl, že se má zítra otec za ním
zastavit. Že mu dá stranou jedno štěně, a že si moc pochutná. Odpověděl jsem
mu, že my psy doma nejíme, ale otci jsem tento vzkaz tlumočil. Nakonec všechno
dobře dopadlo, protože místo psa, přikoulel otec z~pivovaru malý soudek. Měl
z~něj takovou radost, že ten den úplně zapomněl jít do hospody na pivo.

