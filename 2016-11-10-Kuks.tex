% Zdroj: https://www.idnes.cz/zpravy/sto-pohledu/100-pohledu-na-cesko.A161109_130922_ekonomika_jj/diskuse/2
% Datum: 10. 11. 2016 10:24
% Foto: https://1gr.cz/fotky/idnes/16/111/org/JB671660_V1G05.jpg
% Body: +161 -3

Na dobovém snímku lze vidět budovu mosteckých lázní. Konkrétně toto
sanatorium se specializovalo na léčbu pacientů s~respiračními
chorobami. Leckdo by se mohl pozastavit nad faktem, proč léčit nemocné
s~dýchacími obtížemi právě v~jedné z~nejvíce exhalacemi a prachem
znečištěných lokalit. Ovšem tato léčba šokem dosahovala pozoruhodných
výsledků. A~to hlavně u~osob pocházejících z~oblastí s~čistým
ovzduším. Ti kterým se podařilo ozdravný pobyt přežít, se po návratu
domů cítili prokazatelně o~tolik lépe než v~samotných lázních, že se
prakticky nestávalo aby si některý z~nich toužil pobyt na Mostecku
zopakovat. A~to i přesto, že byla léčba plně hrazena c. a k. zdravotní
pojišťovnou. Naopak nulovou úspěšnost měla léčba pacientů
pocházejících z~pánevních oblastí severozápadních Čech a z~některých
okresů severní Moravy. Ti se po návratu do svých domovů cítili stále
stejně blbě. Pod dřevěným mostem, který je na snímku v~popředí,
protékala řeka Bílina. Pacientky zcela nahoře v~podloubí usrkávají
z~pohárků její blahodárnou vodu. Ta byla do kašny za nimi tlačena z~řeky
čerpadly. Její pravidelné užívání dávalo pacientům zapomenout na
jejich dýchací potíže, poněvadž řešili ty střevní. Na schodech i pod
nimi lze spatřit část přeživších pacientů z~jižních Čech. O~ty méně
šťastné se postarala pohřební služba sídlící v~domku vlevo. Ředitele
Graba lze vidět v~okně. Majoritními akcionáři tohoto ústavu byli
vesměs lékaři sanatoria, včetně primáře. Domek vpravo je lázeňský
hostinec Pajšl. V~popředí hostinská Trudi, za ní pak zbytek personálu.
Muž který na Trudi zamilovaně hledí je alkoholik Lojza Rumorád, který
jí pod příslibem manželství připravil nejen o~panenství a o~iluze, ale
i o~veškeré zásoby tvrdého alkoholu. Muž s~plnovousem zcela vlevo je
uhlobaron Sepp Kohle ve společnosti svých synů. Mladšího Hanse a
staršího Kurta. Důvodem jejich přítomnosti je patrně jednání ohledně
dodávek uhlí. Pes kterého drží Hans je zcela jistě Sultán, protože jak
známo Tyrl se nerad fotografoval.

