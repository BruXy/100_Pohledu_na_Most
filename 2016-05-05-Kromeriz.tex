# Datum: 5. 5. 2016 9:53
# Zdroj: https://www.idnes.cz/zpravy/sto-pohledu/100-pohledu-na-cesko.A160505_085937_domaci_jj/diskuse/3
# Foto: https://1gr.cz/fotky/idnes/16/012/org/JB608a67_rizl902p.jpg

Třetí náměstí v Mostě. V tom domě vlevo, bývala hudební škola. Jednou
mě matka odchytila po návratu ze školy a povídá: "Nikam nechod', za
chvíli k nám přijde pan Zářný odnaproti. No ten, co učí na tý hudební
škole na ty nástroje. Pomůže nám na půdě pro tebe nějaký vybrat z
těch, co tam zbyly po našich." Pan Zářný sice dorazil ale z nástrojů
nijak nadšený nebyl. Na flašinet prý nevyučuje, poněvadž klikou prý
může točit kdejaký blbec a ta trumpeta, co se s ní kdysi troubila
půlnoc se také k výuce nehodila. U harmoniky, kterou u nás zapoměl pan
Kozelka, protože umřel, stačily měch prožrat myši a futrál od houslí
byl prázdný. Ty totiž stihnul propít otec, doufaje že se na to
nepřijde. "Nic si z toho nedělejte mladá paní, pošlete kluka zítra za
mnou, já proklepnu na co má talent a nějaký levnější nástroj z druhé
ruky už mu seženeme," konejšil matku pan Zářný. Druhý den mě matka
oblékla do svátečního a vyslala tvořit hudební dějiny. V hudebce jsem
pobyl asi hodinu a když jsem přišel domů, byl tam už pan učitel a
vzrušeně diskutoval s matkou. Pravda, cestou jsem se zapovídal s
Věruškou. Vyprávěl jsem jí, na jaké nástroje jsem hrál a ona mi
ukázala nové punčocháče. Z toho co doma zaslechl, jsem vycítil, že pan
Zářný matku asi moc nepotěšil. Pořád opakoval, že má jenom jedny nervy
a že jsem jim tam rozbil bystu Smetany. To byla asi ta hlava, do níž
jsem vrazil loktem při pokusu o tah smyčcem. Protože u té druhé, co
jsem shodil na chodbě, když jsem se tam klouzal v ponožkách, mě nikdo
neviděl. Také říkal, že co se týče hudebního sluchu, byl by na tom
lépe dělostřelecký kůň, co přežil děvět měsíců kanonády u Verdunu.
Nakonec jí se slzami v očích ukázal prsty, které jsem mu omylem
přicvakl víkem od piána se slovy, že ty ruce ho živí a že ted' může
hrát maximálně na hřeben. Když se ho matka při odchodu nesměle
zeptala, co tedy se mnou, odpověděl: "Víte co, pani, sundejte klukovi
z půdy ten flašinet, kupte mu dřevěnou nohu a on už se nějak
protluče."
